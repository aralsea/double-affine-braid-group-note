\RequirePackage{plautopatch}
\documentclass[uplatex, a4paper, dvipdfmx]{jsarticle}
\usepackage[backend=biber, style=alphabetic, sorting=nyt]{biblatex}
\addbibresource{../bib/double-affine-braid-group.bib} 

\usepackage{amsthm}
\usepackage{amsmath,amsfonts,amssymb}
\usepackage{tikz-cd}
\usepackage{dynkin-diagrams}

\renewcommand{\labelenumi}{(\arabic{enumi})}

\theoremstyle{definition}
\newtheorem{theorem}{定理}[section]
\newtheorem{definition}[theorem]{定義}
\newtheorem{proposition}[theorem]{命題}
\newtheorem{lemma}[theorem]{補題}
\newtheorem{corollary}[theorem]{系}
\newtheorem{fact}[theorem]{事実}
\newtheorem{example}[theorem]{例}
\newtheorem{remark}[theorem]{補足}

\DeclareMathOperator{\Hom}{\mathrm{Hom}}
\DeclareMathOperator{\Tor}{\mathrm{Tor}}
\DeclareMathOperator{\CHom}{\mathcal{H}\!\mathit{om}}
\DeclareMathOperator{\CTor}{\mathcal{T}\!\mathit{or}}
\DeclareMathOperator{\Auteq}{\mathrm{Auteq}}
\DeclareMathOperator{\Cone}{\mathrm{Cone}}
\DeclareMathOperator{\ev}{\mathrm{ev}}
\DeclareMathOperator{\id}{\mathrm{id}}
\DeclareMathOperator{\depth}{\mathrm{depth}}
\DeclareMathOperator{\Pic}{\mathrm{Pic}}
\DeclareMathOperator{\MCG}{\mathrm{MCG}}
\DeclareMathOperator{\PMCG}{\mathrm{PMCG}}
\DeclareMathOperator{\RHom}{\mathrm{RHom}}
\DeclareMathOperator{\Ker}{\mathrm{Ker}}
\DeclareMathOperator{\Image}{\mathrm{Im}}
\DeclareMathOperator{\Aut}{\mathrm{Aut}}
\DeclareMathOperator{\Inn}{\mathrm{Inn}}
\DeclareMathOperator{\Out}{\mathrm{Out}}
\DeclareMathOperator{\Supp}{\mathrm{Supp}}
\DeclareMathOperator{\SL}{\mathrm{SL}}
\DeclareMathOperator{\GL}{\mathrm{GL}}
\DeclareMathOperator{\Spec}{\mathrm{Spec}}
\DeclareMathOperator{\Proj}{\mathrm{Proj}}
\DeclareMathOperator{\Perf}{\mathrm{Perf}}
\DeclareMathOperator{\NS}{\mathrm{NS}}
\DeclareMathOperator{\Ext}{\mathrm{Ext}}
\DeclareMathOperator{\Hilb}{\mathrm{Hilb}}
\DeclareMathOperator{\res}{\mathrm{res}}
\DeclareMathOperator{\Ch}{\mathrm{Ch}}
\DeclareMathOperator{\Coh}{\mathrm{Coh}}
\DeclareMathOperator{\QCoh}{\mathrm{QCoh}}
\DeclareMathOperator{\Ann}{\mathrm{Ann}}
\DeclareMathOperator{\Aff}{\mathrm{Aff}}
\DeclareMathOperator{\FM}{\mathrm{FM}}
\DeclareMathOperator{\red}{\mathrm{red}}
\DeclareMathOperator{\length}{\mathrm{length}}
\DeclareMathOperator{\Mod}{\mathrm{Mod}}
\DeclareMathOperator{\tr}{\mathrm{tr}}
\DeclareMathOperator{\Ob}{\mathrm{Ob}}



\newcommand{\nc}{\newcommand}

%% Calligraphic letters
\nc{\cD}{{\mathcal{D}}}
\nc{\cF}{{\mathcal{F}}}
\nc{\cG}{{\mathcal{G}}}
\nc{\cH}{{\mathcal{H}}}
\nc{\cJ}{{\mathcal{J}}}
\nc{\cM}{{\mathcal{M}}}
\nc{\cO}{{\mathcal{O}}}
\nc{\cU}{{\mathcal{U}}}
\nc{\cW}{{\mathcal{W}}}

%% Blackboard letters
\nc{\bA}{{\mathbb{A}}}
\nc{\bC}{{\mathbb{C}}}
\nc{\bP}{{\mathbb{P}}}
\nc{\bQ}{{\mathbb{Q}}}
\nc{\bR}{{\mathbb{R}}}
\nc{\bZ}{{\mathbb{Z}}}


%% Fraktur letters
\nc{\fg}{{\mathfrak{g}}}
\nc{\fu}{{\mathfrak{u}}}

% hyperref
\usepackage[urlcolor=blue]{hyperref}


\title{半捻り関手の一般論}
\author{荒井 勇人}
\date{\today}
\begin{document}
\maketitle
\section{相対Fourier--Mukai変換}
\begin{definition}
    $T$をNoetherスキーム、$\pi \colon X \to T$を有限型分離的平坦な射とする。
    $\FM_T(X)$を以下で定まる集合とする。
    \begin{equation}
        \FM_T(X) = \{P \in D^b(X \times_T X) \mid P \text{は$\pi_1$-perfectかつ$\pi_2$-perfectで、畳み込みについて可逆}\}/\cong
    \end{equation}
    これは畳み込みを積、$\cO_{\Delta}$を単位元として群になる。
\end{definition}
\begin{proposition}
    $\FM_T(X)$はwell-definedである。
\end{proposition}
\begin{proof}
    畳み込みで閉じていることと、単位元があることを示せばよい。
    $P, Q \in \FM_T(X)$をとる。$P*Q$が$\pi_i$-perfectであることを示す。まず$\pi_{12}, \pi_{23}$は平坦だから$\pi_{12}^*P, \pi_{23}^*Q \in D^b(X \times_T X \times_T X)$である。さらにTor-independent base changeより、$\pi_{12}^*P$は$\pi_{23}$-perfectである。すると\cite[Lemma 5.1]{MR3720794}より$\pi_{12}^*P \otimes \pi_{23}^*Q \in D^b(X \times_T X \times_T X)$である。次にこれが$\pi_1, \pi_3$-perfectであることを示す。そのためには再び\cite[Lemma 5.1]{MR3720794}より任意の$F \in D^b(X)$について$\pi_{12}^*P \otimes \pi_{23}^*Q \otimes \pi_i^*F \in D^b(X \times_T X \times_T X)$であることを見ればよい。$i=3$なら$\pi_{12}^*P$の$\pi_{23}$-perfectnessから、$i=1$なら$\pi_{23}^*Q$の$\pi_{12}$-perfectnessから従う。最後に$P * Q = \pi_{13*}(\pi_{12}^*P \otimes \pi_{23}^*Q)$が$D^b(X\times_T X)$に入っていて$\pi_{i}$-perfectであることを示す。これは\cite[Proposition 2.7]{MR3720794}からわかる。

    単位元は$\cO_\Delta$である。
\end{proof}




\begin{proposition}\label{prop:forget-base}
    $S$をNoetherスキーム、$T \to S$と$X \to T$を有限型分離的平坦な射とする。$\iota \colon X \times_T X \to X \times_S X$とおくと、単射準同型
    \begin{equation}
        \FM_T(X) \to \FM_{S}(X), \quad P \mapsto \iota_*P
    \end{equation}
    がある。
\end{proposition}
\begin{proof}
    まず射影公式より準同型であることがわかる。単射性を示す。
    $\iota \colon X \times_T X \to X \times_S X$は$T \to S$の分離性より閉埋め込みである。$P \in \FM_T (X)$が$\iota_*P \cong \iota_*\cO_\Delta$を満たすと仮定すると、$P$は$D^b(X \times_T X)$の連接層になる。よって$\iota$で押し出す前の時点で$P \cong \cO_\Delta$である。
\end{proof}


\begin{proposition}
    $T$を(局所)Noetherスキーム、$T' \subset T$を正則閉埋め込み、$X \to T$を平坦射、$X' = X \times_T T'$とする。このとき以下はTor-independent base changeである。
    \[
        \begin{tikzcd}
            X' \ar[r]\ar[d, "\Delta_{X'}"] &X \ar[d, "\Delta_X"]\\
            X' \times_{T'} X' \ar[r] & X \times_T X
        \end{tikzcd}
    \]
\end{proposition}
\begin{proof}
    \[
        \begin{tikzcd}
            X' \ar[r]\ar[d, "\Delta_{X'}"] &X \ar[d, "\Delta_X"]\\
            X' \times_{T'} X' \ar[r] \ar[d] & X \times_T X \ar[d]\\
            T' \ar[r] & T\\
        \end{tikzcd}
    \]
    $T = \Spec R, T' = \Spec (R/I), X = \Spec A$としてよい。$i > 0$に対して
    \begin{equation}
        \Tor^{A \otimes_R A}_i((A \otimes_R A) \otimes_R (R/I), A) = 0
    \end{equation}
    を示す。
    $T' \subset T$は正則埋め込みだから、正則列$(f_1, \dots, f_r)$で$I = (f_1, \dots, f_r)$となるものが存在する。この正則列に付随するKoszul複体$K_{\bullet}(f_1, \dots, f_r; R)$は、$A \otimes_R A$が$R$上平坦だから$(A \otimes_R A) \otimes_R (R/I)$の$A \otimes_R A$加群としての自由分解$K_{\bullet}(f_1, \dots, f_r; A \otimes_R A) \cong K_{\bullet}(f_1, \dots, f_r; R) \otimes_R (A \otimes_R A)$を誘導する。よって
    \begin{align}
        \Tor^{A \otimes_R A}_i((A \otimes_R A) \otimes_R (R/I), A) & \cong H^{-i}(K_{\bullet}(f_1, \dots, f_r; A \otimes_R A) \otimes_{A \otimes_R A} A)               \\
                                                                   & \cong H^{-i}(K_{\bullet}(f_1, \dots, f_r; R) \otimes_R (A \otimes_R A) \otimes_{A \otimes_R A} A) \\
                                                                   & \cong H^{-i}(K_{\bullet}(f_1, \dots, f_r; R) \otimes_R A)
    \end{align}
    となるが、$A$は$R$上平坦だから右辺は$i > 0$について$0$である。
\end{proof}
\begin{proposition}
    $T$をNoetherスキーム、$\pi \colon X \to T$を有限型分離的平坦な射とする。$\varphi \colon T' \to T$をNoetherスキームの射とし、$\psi \colon X' = X \times_T T \to X$とする。さらに以下がTor-independent base changeだと仮定する(例えば$\varphi$が平坦ならよい)。
    \[
        \begin{tikzcd}
            X' \ar[r] \ar[d, "\Delta_{X'}"'] & X \ar[d, "\Delta_X"]\\
            X' \times_{T'} X' \ar[r] & X \times_T X
        \end{tikzcd}
    \]
    このとき群準同型
    \begin{equation}
        \FM_T(X) \to \FM_{T'}(X'), \quad P \mapsto \psi^*P
    \end{equation}
    がある。
\end{proposition}
\begin{proof}
    平坦性とTor-independent base change からわかる。
\end{proof}
\section{捻り関手}
\begin{proposition}\label{prop:kernel-to-functor}
    $X$を$k$上の準射影的スキームとする。このとき単射準同型
    \begin{equation}
        \FM_k(X) \to \Auteq D^b(X), \quad P \mapsto \Phi_P
    \end{equation}
    がある。
\end{proposition}
\begin{proof}
    $X$が非特異射影的ならばOrlovの結果である。
    準射影的な場合は\cite{MR2434186}と\cite{MR3556457}の結果を使う。

    まずwell-definednessを示す。群準同型
    \begin{equation}
        \FM_k(X) \to \Auteq D_{\QCoh}(\cO_X), \quad P \mapsto \Phi_P
    \end{equation}
    がwell-definedなので$\Phi_P$が$D^b(X)$を保つことを示せばよいが、\cite[Lemma 6.11, Proposition 6.12]{MR2434186}より$D^b(X) \subset D_{\QCoh}(\cO_X)$は$k$線形な三角圏同値で不変だからよい。

    次に単射性を示す。
    \cite[Remark 1.3, Theorem 1.5]{MR3556457}によると、$X$と$Y$が$k$上準射影的スキームで$E, E' \in D(\QCoh(X \times_k Y))$が$\Phi_E \cong \Phi_{E'} = \Phi \colon \Perf(X) \to D(\QCoh(Y))$を満たし、さらに任意の$m, n$について$\Hom^{<0}(\Phi(\cO_X(m)), \Phi(\cO_X(n))) = 0$ならば、$E \cong E'$である。
    $P, P' \in \FM_k(X)$が$\Phi_P \cong \Phi_{P'} \in \Auteq D^b(X)$を満たすとする。このとき$\Phi_P, \Phi_{P'}$を関手$\Perf(X) \to D(\QCoh(X))$とみなしたものも自然同型で、仮定より$\Hom^{<0}(\Phi_P(\cO_X(m)), \Phi_P(\cO_X(n))) = 0$である。よって$P \cong P'$である。
\end{proof}
\begin{remark}
    $X$が$k$上の準射影的スキームのとき、$\FM_k(X)$は以下のようにかける。
    \begin{equation}
        \FM_k(X) = \{P \in D^b(X \times_k X) \mid P \text{は$\pi_1$-perfectかつ$\pi_2$-perfectで、$\Phi_P$は圏同値$D^b(X) \to D^b(X)$を誘導する}\}/\cong
    \end{equation}
    右辺の元$P$が畳み込みについて可逆であることを示せばよい。\cite{MR3720794}より、$\Phi_P$の右随伴は$P_R = \CHom(P, \pi_1^!\cO_X)$で与えられ、仮定より$\Phi_{P * P_R} \cong \Phi_P \circ \Phi_{P_R} \cong \id \cong \Phi_{\cO_{\Delta}}$である。よって命題\ref{prop:kernel-to-functor}の証明より$P * P_R \cong \cO_{\Delta}$である。同様に$P_L * P \cong \cO_{\Delta}$でもあるから、$P$は畳み込みについて可逆である。
\end{remark}

\begin{definition}
    $X$を$k$上の非特異準射影的多様体とする。$E \in D^b(X)$が以下の条件を満たすとき、球面対象という。
    \begin{enumerate}
        \item $E$はproper supportを持つ
        \item $E \otimes \omega_X \cong E$
        \item $\Hom^*(E, E) \cong k \oplus k[-\dim X]$
    \end{enumerate}
\end{definition}


\begin{proposition}
    $X, T$を$k$上分離的かつqcなスキーム、$X \to T$を平坦射とし、$T$はNoetherかつ正則で$n$次元とする。$i \colon X_0 \to X$を閉点$0 \in T$のファイバーとし、$F \in D_{\QCoh}^b(\cO_{X_0})$を$\Hom^{<0}(i_*F, i_*F) = 0$を満たす対象とする。このとき
    \begin{enumerate}
        \item $\Hom^{<0}(F, F) = 0$
        \item $i_* \colon \Hom(F, F) \to \Hom(i_*F, i_*F)$は同型
    \end{enumerate}
    である。
\end{proposition}
\begin{proof}
    $T$は$0$を含むように小さくとりかえてよい。そこで以下のような$\{0\} = T_0 \subset T_1 \subset \cdots \subset T_n = T$が存在するように$T$を小さくとりかえる。
    \begin{itemize}
        \item $\dim T_j = j$
        \item $T_j$は正則
        \item $T_{j-1}$の$T_j$の中での法束は自明
    \end{itemize}
    具体的には$\cO_{T, 0}$の正則巴系で定まる部分スキームの列をとればよい。
    \[
        \begin{tikzcd}
            X_0 \ar[r, "i_1"] \ar[d] & X_1 \ar[r, "i_2"] \ar[d] & \cdots \ar[r, "i_n"]& X_n \ar[d] \\
            T_0 \ar[r] & T_1 \ar[r] & \cdots \ar[r]& T_n
        \end{tikzcd}
    \]
    $T_j$のファイバーを$X_j$とし、$i_j \colon X_{j-1} \to X_{j}, F_{j} = i_{j*}F_{j-1} \in D_{\QCoh}^b(\cO_{X_j})$とおく。このとき任意の$j = 0, \dots, n-1$について$\Hom^{\leq 0}(F_{j}, F_{j}) \to \Hom^{\leq 0}(F_{j+1}, F_{j+1})$が同型であることを示せばよい。
    \[
        \begin{tikzcd}
            X_{j} \ar[r, "i_{j+1}"] \ar[d] & X_{j+1} \ar[d]\\
            T_{j} \ar[r] & T_{j+1}
        \end{tikzcd}
    \]
    $T_j$たちの取り方より$\cO_{T_{j+1}}(-T_{j})$は自明束だから、それを引き戻した$\cO_{X_{j+1}}(-X_{j})$も自明束である。よって命題\ref{prop:triangle-for-push-pull}より、$D_{\QCoh}(\cO_{X_j})$での完全三角形
    \begin{equation}
        F_j[1] \to i_{j+1}^*i_{j+1, *}F_j \to F_j \xrightarrow{+1}
    \end{equation}
    がある。これに$\Hom(-, F_j)$を適用した長完全列をとると、
    \[
        \begin{tikzcd}
            \Hom(F_j[1], F_j) & \Hom(F_{j+1}, F_{j+1}) \ar[l] & \Hom(F_j, F_j) \ar[l] \\
            \Hom(F_j[2], F_j) \ar[rru] & \Hom(F_{j+1}[1], F_{j+1}) \ar[l] & \Hom(F_j[1], F_j) \ar[l] \\
            \Hom(F_j[3], F_j) \ar[rru] & \Hom(F_{j+1}[2], F_{j+1}) \ar[l] & \Hom(F_j[2], F_j) \ar[l] \\
            \Hom(F_j[4], F_j) \ar[rru]& \cdots &\\
        \end{tikzcd}
    \]
    となる。数学的帰納法より$\Hom^{<0}(F_{j+1}, F_{j+1}) = 0$としてよく、すると長完全列より
    \begin{align}
        \Hom^{-1}(F_j, F_j) & = \Hom^{-3}(F_j, F_j) = \Hom^{-5}(F_j, F_j) = \cdots \\
        \Hom^{-2}(F_j, F_j) & = \Hom^{-4}(F_j, F_j) =\Hom^{-6}(F_j, F_j) = \cdots
    \end{align}
    がわかる。さらに$F_j$は有界な複体だから、十分小さい$m$について$\Hom^m(F_j, F_j) = 0$となる。これらより$\Hom^{<0}(F_j, F_j) = 0$がわかり、すると長完全列より$\Hom(F_j, F_j) \to \Hom(F_{j+1}, F_{j+1})$が同型となる。
\end{proof}

\subsection{体上有限型の場合}
$X \to T$を代数閉体$k$上の非特異多様体の間の平坦射とし、$X$は準射影的とする。また$0 \in T$を閉点とし、$X_0$をそのファイバーとする。このとき以下の群準同型がある。

\[
    \begin{tikzcd}
        \FM_k(X_0) \ar[d, hookrightarrow]& \FM_T(X) \ar[l] \ar[r, hookrightarrow]& \FM_k(X) \ar[d, hookrightarrow]\\
        \Auteq D^b(X_0)& & \Auteq D^b(X)
    \end{tikzcd}
\]
ファイバーから来る球面対象の定める捻り関手が$\FM_T(X)$に入っていることを示す。
\begin{proposition}
    $X, T$を$k$上の非特異代数多様体、$X \to T$を平坦射とする。$i \colon X_0 \to X$を閉点$0 \in T$のファイバーとし、$E \in D^b(X)$とする。

    \[
        \begin{tikzcd}
            X_0 \arrow[r,"i"]\arrow[d, "\Delta_{X_0}"] & X \arrow[d,"d"]\arrow[rdd, bend left, "\Delta_X"]&\\
            X_0 \times X_0 \arrow[r, "k"]\arrow[rrd, bend right, "j=i \times i"]& X \times_T X\arrow[rd, "\iota"]&\\
            & & X \times X
        \end{tikzcd}
    \]
    このとき、$i_*E$に付随する捻り関手$T_{i_*E}$の積分核
    \begin{equation}
        P_{i_*E} = \Cone((i_*E)^\vee \boxtimes (i_*E) \to \Delta_{X*}\cO_X)
    \end{equation}
    は、
    \begin{equation}
        P_{E} = \Cone(k_*(E' \boxtimes E)\to d_*\cO_X)
    \end{equation}
    の$\iota$による押し出しと同型であり、$P_E$は$\pi_1, \pi_2$-perfectである。
    さらに$i_*E$が球面対象なら、$P_E \in \FM_T(X)$である。
\end{proposition}
\begin{proof}
    前半は\cite{2023arXiv230212501A}の議論からわかる。
    $\pi_1, \pi_2$-perfectであることを示す。two-out-of-threeより$k_*(E' \boxtimes E)$が$\pi_1, \pi_2$-perfectであることを示せば十分である。

    \[
        \begin{tikzcd}
            X_0 \times X_0 \arrow[r, "k"]\arrow[d, "p_1"] & X \times_T X\arrow[d, "\pi_1"]\\
            X_0 \arrow[r, "i"] & X
        \end{tikzcd}
    \]
    \cite[Proposition 2.7]{MR3720794}より、$E' \boxtimes E$が$\pi_1 \circ k$-perfectであることを示せば$k_*(E' \times E)$が$\pi_1$-perfectなことがわかる。$F \in D^b(X)$とすると
    \begin{align}
        (\pi_1 \circ k)^*F \otimes (E' \boxtimes E) & \cong (i^*F \otimes E') \boxtimes E
    \end{align}
    で、これが$D^b(X_0 \times X_0)$に入ることを見ればよい。$j$で押し出すとKunneth formulaより$i_*(i^*F \otimes E') \boxtimes i_*E \cong (F \otimes i_*(E'))\boxtimes i_*E$となるから$X$の非特異性より$D^b(X \times_T X)$に入る。$j$は閉埋め込みだから元の$(i^*F \otimes E') \boxtimes E$も$D^b(X_0 \times X_0)$に入り、これは$E' \boxtimes E$が$\pi_1 \circ k$-perfectであることを意味する。$\pi_2$についても同様。

    さらに$i_*E$が球面対象なら、命題\ref{prop:forget-base}の証明より$P_E$は畳み込みについて可逆である。
\end{proof}

\section{Appendix}
\subsection{記法}
\begin{itemize}
    \item 導来関手の$R$と$L$は原則省略する。
    \item $X$が環付き空間のとき$\Mod(\cO_X)$を$\cO_X$加群の圏、$D(\cO_X)$を$\Mod(\cO_X)$の導来圏とする。$+, -, b$をつけて有界導来圏を表す。
    \item $X$がスキームのとき、$D_{\QCoh}(\cO_X) \subset D(\cO_X)$をコホモロジーが凖連接層であるような複体のなす充満部分圏とする。
    \item $X$がNoetherスキームのとき、$D_{\Coh}(\cO_X) \subset D(\cO_X)$をコホモロジーが連接層であるような複体のなす充満部分圏とする。また$D^b(X) = D_{\Coh}^b(\cO_X)$とする。($X$がNoetherなのでこれは連接層の圏$\Coh(X)$の導来圏と同値である。)
\end{itemize}

\subsection{導来関手}
\begin{definition}
    $X$を環付き空間とし、$E \in D(\cO_X)$とする。このとき$E$の双対$E^\vee$を
    \begin{equation}
        E^\vee = \RHom(E, \cO_X)
    \end{equation}
    により定義する。
\end{definition}
\begin{proposition}
    $f \colon X \to Y$を環付き空間の射とし、$K, M \in D(\cO_Y)$とする。このとき自然な射
    \begin{equation}
        f^*\CHom(K, M) \to \CHom(f^*K, f^*M)
    \end{equation}
    は以下のいずれかの状況で同型である。
    \begin{enumerate}
        \item $K$がperfect
        \item $X, Y$がNoetherスキーム、$f$が平坦で、$K \in D_{\Coh}^-(\cO_Y), M \in D^+(\cO_Y)$
    \end{enumerate}
\end{proposition}
\begin{proof}
    {\cite[\href{https://stacks.math.columbia.edu/tag/0GM7}{Tag 0GM7}]{stacks-project}}より直ちに従う。
\end{proof}
\begin{corollary}\label{cor:pullback-and-dual}
    $f \colon X \to Y$をNoetherスキームの間の平坦射とすると、任意の$E \in D_{\Coh}^-(\cO_Y)$について双対は引き戻しと交換する。つまり$f^*(E^\vee) \cong (f^*E)^\vee$である。
\end{corollary}
\begin{proposition}\label{prop:tensor-and-dual}
    $X$を環付き空間とし、$L, E \in D(\cO_X)$とする。さらに$L$がperfectであるとする。このとき自然な同型
    \begin{equation}
        E^\vee \otimes L^\vee \cong (E \otimes L)^\vee
    \end{equation}
    が存在する。
\end{proposition}
\begin{proof}
    {\cite[\href{https://stacks.math.columbia.edu/tag/0G40}{Tag 0G40}]{stacks-project}}より自然な同型
    \begin{equation}
        \CHom(E, \cO_X) \otimes L^\vee \cong \CHom(\CHom(L^\vee, E), \cO_X)
    \end{equation}
    がある。
    特に$L^{\vee\vee} \cong L$もわかる。
    そして {\cite[\href{https://stacks.math.columbia.edu/tag/08DQ}{Tag 08DQ}]{stacks-project}}より
    \begin{equation}
        \CHom(L^\vee, E) \cong E \otimes L^{\vee\vee} \cong E \otimes L
    \end{equation}
    なので命題が成り立つ。
\end{proof}
\subsection{Grothendieck--Verdier双対}
\begin{theorem}[{\cite[\href{https://stacks.math.columbia.edu/tag/0A9E}{Tag 0A9E}]{stacks-project}}]
    $f \colon X \to Y$をqcqsスキームの間の射とする。このとき
    \begin{equation}
        f_* \colon D_{\QCoh}(\cO_X) \to D_{\QCoh}(\cO_Y)
    \end{equation}
    は右随伴$a \colon D_{\QCoh}(\cO_Y) \to D_{\QCoh}(\cO_X)$を持つ。さらにこれは$D_{\QCoh}^+$を保つ。
\end{theorem}
\begin{theorem}[{\cite[\href{https://stacks.math.columbia.edu/tag/0GEW}{Tag 0GEW}]{stacks-project}}]\label{thm:Grothendieck-Verdier-duality-for-complexes}
    $f \colon X \to Y$をNoetherスキームの間の固有射とする。このとき任意の$L \in D_{\Coh}^-(\cO_X), K \in D_{\QCoh}^+(\cO_Y)$に対して自然な同型
    \begin{equation}
        f_* \CHom(L, a(K)) \xrightarrow{\sim} \CHom(f_*L, K)
    \end{equation}
    がある。
\end{theorem}
\begin{definition}[{\cite[\href{https://stacks.math.columbia.edu/tag/0A74}{Tag 0A74}]{stacks-project}}]
    $i \colon Z \to X$をスキームの閉埋め込みとする。$\cO_X$加群$F$に対して、層$\CHom_{\cO_X}(i_*\cO_Z, F)$は自然に$\cO_Z$加群とみなせる。
    これにより定まる関手$\Mod(\cO_X) \to \Mod(\cO_Z)$を$\CHom(\cO_Z, -)$と書く。
\end{definition}
\begin{proposition}[{\cite[\href{https://stacks.math.columbia.edu/tag/0A74}{Tag 0A74}]{stacks-project}}]\label{prop:Grothendieck-Verdier-duality-for-closed-immersion}
    $i \colon Z \to X$をスキームの閉埋め込みとする。このとき
    \begin{enumerate}
        \item 関手$\CHom(\cO_Z, -) \colon \Mod(\cO_X) \to \Mod(\cO_Z)$は左完全である。
        \item その右導来関手を$R\CHom(\cO_Z, -) \colon D(\cO_X) \to D(\cO_Z)$とおくと、これは$(L i_* = )i_* = R i_* \colon D(\cO_Z) \to D(\cO_X)$の右随伴である。
        \item $R\CHom(\cO_Z, -)$は$D_{\QCoh}^+$を保つ。
        \item $i_* \colon D_{\QCoh}(\cO_Z) \to D_{\QCoh}(\cO_X)$の右随伴$a$と$R\CHom(\cO_Z, -)$は$D_{\QCoh}^+$へ制限したときに一致する。
    \end{enumerate}
\end{proposition}
\subsection{射影公式}

\begin{proposition}[{\cite[\href{https://stacks.math.columbia.edu/tag/08EU}{Tag 08EU}]{stacks-project}}]\label{prop:projection-formula-for-qcoh}
    $f \colon X \to Y$をスキームの間のqcqs射とする。$E \in D_{\QCoh}(\cO_X), K \in D_{\QCoh}(\cO_Y)$とする。このとき自然な射
    \begin{equation}
        f_*E \otimes K \to f_*(E \otimes f^*K)
    \end{equation}
    は同型である。
\end{proposition}
\begin{proposition}\label{prop:projection-formula-for-modules}
    $f \colon X \to Y$を環付き空間の射とし、$E \in D(\cO_X), K \in D(\cO_Y)$とする。このとき自然な射
    \begin{equation}
        f_*E \otimes K \to f_*(E \otimes f^*K)
    \end{equation}
    は以下の状況で同型である。
    \begin{enumerate}
        \item $K$がperfect({\cite[\href{https://stacks.math.columbia.edu/tag/0B54}{Tag 0B54}]{stacks-project}})
        \item $f$が閉集合への同相({\cite[\href{https://stacks.math.columbia.edu/tag/0B55}{Tag 0B55}]{stacks-project}})
    \end{enumerate}
\end{proposition}

\subsection{Fourier--Mukai変換}
$X, Y$を$S$上のスキームとし、$K \in D(\cO_{X \times_S Y})$とする。
\begin{equation}
    \Phi_K = p_{Y*}(p_{X,*}(-)\otimes_{\cO_{X \times_S Y}} K) \colon D(\cO_X) \to D(\cO_Y)
\end{equation}
を$K$に付随するFourier--向井関手という。$D_{\QCoh}$などの部分圏の間の関手になっているとき、それもFourier--向井関手という。

\begin{lemma}[{\cite[\href{https://stacks.math.columbia.edu/tag/0FYR}{Tag 0FYR}]{stacks-project}}]
    $X \to S$がqcqs射で$K \in D_{\QCoh}$なら$\Phi_K \colon D_{\QCoh}(\cO_X) \to D_{\QCoh}(\cO_Y)$である。
\end{lemma}
\begin{lemma}[{\cite[\href{https://stacks.math.columbia.edu/tag/0FYS}{Tag 0FYS}]{stacks-project}}]
    $X, Y, Z$が$S$上qcqsなスキームで、$X$と$Z$が$S$上tor independentかつ$Y$が$S$上平坦だとする。$K, K' \in D_{\QCoh}$のとき
    \begin{equation}
        K'' = p_{XZ,*}(p_{XY}^*K \otimes p_{YZ}^*K'')
    \end{equation}
    とおくと、
    \begin{equation}
        \Phi_{K'} \circ \Phi_K = \Phi_{K''}\colon D_{\QCoh}(\cO_X) \to D_{\QCoh}(\cO_Z)
    \end{equation}
    である。
\end{lemma}
\begin{example}\label{ex:composition-of-Fourier-Mukai-over-field}
    $X, Y, Z$が$S$上qcqsで平坦なら使える。
\end{example}
\begin{example}\label{ex:push-pull-as-Fourier-Mukai}
    $f \colon X \to Y$を$S$上分離的なスキームの間の射とする。このとき以下で定まる$f$のグラフ$\Gamma_f \colon X \to X \times_S Y$は閉埋め込みである。
    \[
        \begin{tikzcd}
            X \ar[rd, "\Gamma_f"] \ar[rdd, bend right, "\id_X"] \ar[rrd, bend left, "f"]& & \\
            & X \times_S Y \ar[r] \ar[d] & Y \ar[d]\\
            & X            \ar[r]        & S
        \end{tikzcd}
    \]
    そして
    \begin{align}
        \Phi_{\cO_{\Gamma_f}}^{X \to Y} = f_* & \colon D(\cO_X) \to D(\cO_Y) \\
        \Phi_{\cO_{\Gamma_f}}^{Y \to X} = f_* & \colon D(\cO_Y) \to D(\cO_X)
    \end{align}
    である。
\end{example}
\begin{proof}
    $\Gamma_f$は閉埋め込みなので、\ref{prop:projection-formula-for-qcoh}より射影公式が使える。
    よって
    \begin{align}
        \Phi_{\cO_{\Gamma_f}}^{X \to Y}(-) & = p_{Y*}(p_X^*(-) \otimes \cO_{\Gamma_f})               \\
                                           & = p_{Y*}(p_X^*(-) \otimes \Gamma_{f*}\cO_X)             \\
                                           & = p_{Y*}\Gamma_{f*}(\Gamma_{f}^*p_X^*(-) \otimes \cO_X) \\
                                           & = f_*(-)
    \end{align}
    \begin{align}
        \Phi_{\cO_{\Gamma_f}}^{X \to Y}(-) & = p_{X*}(p_Y^*(-) \otimes \cO_{\Gamma_f})               \\
                                           & = p_{X*}(p_Y^*(-) \otimes \Gamma_{f*}\cO_X)             \\
                                           & = p_{X*}\Gamma_{f*}(\Gamma_{f}^*p_Y^*(-) \otimes \cO_X) \\
                                           & = f^*(-)
    \end{align}
    となる。
\end{proof}
\begin{example}\label{ex:line-bundle-as-Fourier-Mukai}
    $X$を$S$上分離的なスキームとし、$L$を$X$上の可逆層とする。このとき
    \begin{equation}
        \Phi_{\Delta_*L}(-) = L \otimes f^*(-) \colon D(\cO_X) \to D(\cO_Y)
    \end{equation}
    である。
\end{example}
\begin{proof}
    $\Delta$は閉埋め込みなので、\ref{prop:projection-formula-for-modules}より射影公式が使える。
    よって
    \begin{align}
        \Phi_{\Delta_*L}(-) & = p_{Y*}(p_X^*(-) \otimes \Delta_* L)        \\
                            & = p_{Y*}\Delta_*(\Delta^*p_X^*(-) \otimes L) \\
                            & = (-) \otimes L
    \end{align}
    である。
\end{proof}

\subsection{Cartier因子}
\begin{definition}
    $X$をスキームとする。閉部分スキーム$D \subset X$が有効Cartier因子であるとは、$D$の定義イデアル$I$が可逆層であることをいう。
    このとき$\cO_X(-D) = I, \cO_X(D) = \cO_X(-D)^{-1}$と定める。
    定義から完全列
    \begin{equation}
        0 \to \cO_X(-D) \to \cO_X \to \cO_D \to 0
    \end{equation}
    が存在する。

\end{definition}
\begin{proposition}
    $X$をスキームとし、$D \subset X$を閉部分スキームとする。このとき以下は同値である。
    \begin{enumerate}
        \item $D$は有効Cartier因子である。
        \item $D$の定義イデアルは局所的に非零因子で生成される。
    \end{enumerate}
\end{proposition}
\begin{proposition}\label{prop:duality-for-Cartier-divisor}
    $X$をスキームとし、$i \colon D\hookrightarrow X$を有効Cartier因子とする。
    このとき関手
    \begin{equation}
        i^*(-)\otimes \cO_D(D)[-1] \colon D(\cO_X) \to D(\cO_D)
    \end{equation}
    は関手$i_* \colon D(\cO_D) \to D(\cO_X)$の右随伴である。
    さらにこれは$D_{\QCoh}$を保ち、
    特に$X$がqcqsスキームなら($D$もqcqsで)$i_* \colon D_{\QCoh}(\cO_D) \to D_{\QCoh}(\cO_X)$の右随伴$i^\times$と一致する。
\end{proposition}
\begin{proof}
    まず{\cite[\href{https://stacks.math.columbia.edu/tag/0AA4}{Tag 0AA4}]{stacks-project}}より$i^*(-)\otimes \cO_D(D)[-1] \cong R\CHom(\cO_D, -)$だから、命題\ref{prop:Grothendieck-Verdier-duality-for-closed-immersion}より$i^*(-)\otimes \cO_D(D)[-1]$は$i_*$の右随伴である。
    さらに$i^*(-)\otimes \cO_D(D)[-1]$は$D_{\QCoh}$を保つ。
    よって誘導される関手$i^*(-)\otimes \cO_D(D)[-1] \colon D_{\QCoh}(\cO_X) \to D_{\QCoh}(\cO_D)$も$i_*$の右随伴であり、特に$X$がqcqsスキームなら$i^\times$と一致する。
\end{proof}

\begin{proposition}\label{prop:triangle-for-push-pull}
    $k$を体とし、$X$を$k$上分離的かつqcなスキーム、$i \colon D \hookrightarrow X$を有効Cartier因子とする。
    このとき任意の$F \in D_{\QCoh}(\cO_D)$に対して、$D_{\QCoh}(\cO_Y)$の完全三角形
    \begin{equation}
        F \otimes \cO_D(-D)[1] \to i^*i_*F \to F \xrightarrow{+1}
    \end{equation}
    が存在する。ここで$i^*i_*F \to F$は随伴$i^* \dashv i_*$の余単位射、$F \otimes \cO_D(-D)[1] \to i^*i_*F$は
    \begin{equation}
        \Hom(F \otimes \cO_D(-D)[1], i^*i_*F) \cong \Hom(F, i^*i_*F\otimes \cO_D(D)[-1]) = \Hom(F, i^\times i_*F)
    \end{equation}
    によって随伴$i_* \dashv i^\times$の単位射に対応するものである。

    $F = \cO_D$のときはより強く、
    \begin{equation}
        i^*i_*\cO_D \cong \cO_D \oplus \cO_D(-D)[-1]
    \end{equation}
    が成り立つ。
\end{proposition}
\begin{proof}
    \cite[corollary 11.4]{MR2244106}の証明がこの場合に使えることを確かめる。

    まず$F = \cO_D$のときに示す。
    完全列
    \begin{equation}\label{eq:koszul-resolution}
        0 \to \cO_X(-D) \to \cO_X \to i_*\cO_D \to 0
    \end{equation}
    が$i_*\cO_D$の平坦分解を与えるから、これにより$i^*$を計算すればよい。

    次に$F$が(一般の複体ではなく)準連接層のときに成立することを示す。
    射影公式よりと分解\eqref{eq:koszul-resolution}より
    \begin{align}
        i_*i^*i_*F & \cong i_*\cO_D \otimes i_*F                                     \\
                   & \cong (0 \to i_*F\otimes\cO_X(-D) \xrightarrow{d} i_*F \to 0)   \\
                   & \cong (0 \to i_*(F\otimes\cO_D(-D)) \xrightarrow{d} i_*F \to 0)
    \end{align}
    となり、微分$d$は$D$に制限されているので$0$となる。よって$i^*i_*F$のコホモロジーは($i_*$してから計算できるので)
    \[
        \cH^j(i^*i_*F) \cong \begin{cases}
            F                   & (j = 0)            \\
            F \otimes \cO_D(-D) & (j = -1)           \\
            0                   & (\text{otherwise})
        \end{cases}
    \]
    となる。
    特に$i^*i_*F$は2項複体だから、完全三角形
    \begin{equation}
        \cH^{-1}[1] \to i^*i_*F \to \cH^0 \xrightarrow{+1}
    \end{equation}
    が存在する。
    あとは
    右端のコホモロジーへの自然な射$i^*i_*F \to \cH^0 \cong F$が随伴$i^*\dashv i_*$の余単位射に、
    左端のコホモロジーからの自然な射$F \otimes \cO_D(-D)[1] \cong \cH^{-1}[1] \to i^*i_*F$が随伴$i_* \dashv i^\times$の単位射に
    対応することを示せばよい。
    $P^\bullet \to i_*F$と$i_*F \to I^\bullet$をそれぞれ$i_*F$の平坦分解と単射分解とする。
    このとき$\id_{i_*F}$は同型
    \begin{equation}
        \Hom(i_*F, i_*F) \cong \Hom(P^\bullet, i_*F) \cong \Hom(i^*P^\bullet, F)
    \end{equation}
    を通して$P^\bullet \to i_*F$の随伴
    \[
        \begin{tikzcd}
            \cdots \ar[r] & i^*P^{-1} \ar[r] \ar[d] & i^*P^0 \ar[r] \ar[d] & 0 \ar[d]\ar[r]& \cdots \\
            \cdots \ar[r] & 0 \ar[r]     & F \ar[r]     & 0\ar[r]& \cdots
        \end{tikzcd}
    \]
    に対応する。これは右端のコホモロジーへの自然な射に他ならない。
    同様に同型
    \begin{equation}
        \Hom(i_*F, i_*F) \cong \Hom(i_*F, I^\bullet) \cong \Hom(F, a(I^\bullet)) \cong \Hom(F, R\CHom(\cO_D, I^\bullet))
    \end{equation}
    により$\id_{i_*F}$は
    \[
        \begin{tikzcd}
            \cdots \ar[r] & 0 \ar[r]  \ar[d]    & F \ar[r]\ar[d]     & 0\ar[r]\ar[d]& \cdots\\
            \cdots \ar[r] & 0 \ar[r]  & R\CHom(\cO_D, I^0) \ar[r] & R\CHom(\cO_D, I^1)\ar[r]& \cdots
        \end{tikzcd}
    \]
    に対応し、これは左端のコホモロジーからの自然な射に他ならない。

    最後に一般の$F \in D_{\QCoh}(\cO_D)$に対して示す。以下では$\times_k$を$\times$と略記する。
    $i \colon D \hookrightarrow X$のグラフを$\Gamma \subset D \times X$、その転置を$\overline{\Gamma} \subset X \times D$とする。
    \[
        \begin{tikzcd}
            && D &\\
            &\Gamma \times D \ar[r, hook, "\varphi"]\ar[d, "\pi"] &D \times D \times D \ar[d, hook, "\psi"] \ar[u, "\pi_2"] & D \times \overline{\Gamma} \ar[l, hook', "\overline{\varphi}"]\\
            D  \ar[r, hook, "\Delta"] \ar[ru, bend left, "\eta"]&D \times D & D \times X \times D \ar[l, "\pi_{DD}"] &
        \end{tikzcd}
    \]
    という可換図式を考える。ただし$\eta$は$D \xrightarrow{\sim} \Gamma$と$\id_D$の積である。
    有効Cartier因子$\varphi \colon D \times D \times D \hookrightarrow D \times X \times D$と$D \times D \times D$上の準連接層$\overline{\varphi}_*\cO_{D \times \overline{\Gamma}}$に対して命題を適用すると、完全三角形
    \begin{equation}\label{eq:triangle-on-DDD}
        \overline{\varphi}_* \cO_{D \times \overline{\Gamma}}\otimes \pi_2^*\cO_D(-D)[1]\to \psi^*\psi_*\overline{\varphi}_*\cO_{D \times \overline{\Gamma}}\to \overline{\varphi}_*\cO_{D \times \overline{\Gamma}} \xrightarrow{+1}
    \end{equation}
    を得る。
    $\eta \colon D \hookrightarrow D \times D \times D$を対角射とすると、
    \begin{equation}
        \varphi^*\overline{\varphi}_*\cO_{X \times \overline{\Gamma}} \cong \eta_*\cO_D
    \end{equation}
    かつ
    \begin{equation}
        \varphi^*(\overline{\varphi}_* \cO_{D \times \overline{\Gamma}}\otimes \pi_2^*\cO_D(-D)[1]) \cong \eta_*\cO_D(-D)[1]
    \end{equation}
    だから、\eqref{eq:triangle-on-DDD}を$\varphi$で引き戻すことで$D_{\QCoh}(\cO_{\Gamma \times D})$の完全三角形
    \begin{equation}
        \eta_*\cO_D(-D)[1] \to \varphi^*\psi^*\psi_*\overline{\varphi}_*\cO_{D \times \overline{\Gamma}} \to \eta_*\cO_D \xrightarrow{+1}
    \end{equation}
    を得る。
    これをさらに$\pi$で押し出すと
    \begin{equation}
        \Delta_*\cO_D(-D)[1] \to \pi_*\varphi^*\psi^*\psi_*\overline{\varphi}_*\cO_{D \times \overline{\Gamma}} \to \Delta_*\cO_D \xrightarrow{+1}
    \end{equation}
    という$D_{\QCoh}(\cO_{D \times D})$の完全三角形となる。
    ここで
    \begin{align}
        \pi_*\varphi^*\psi^*\psi_*\overline{\varphi}_*\cO_{D \times \overline{\Gamma}} & \cong \pi_{DD}((\psi \circ \varphi)_*\cO_{\Gamma \times D} \otimes (\psi \circ \overline{\varphi})_*\cO_{D \times \overline{\Gamma}}) \\
                                                                                       & \cong \pi_{DD}(\pi_{DX}^*\cO_{\Gamma} \otimes \pi_{XD}\cO_{\overline{\Gamma}})
    \end{align}
    が成り立つ。1つ目の同型は$\psi \circ \varphi$についての射影公式(命題\ref{prop:projection-formula-for-modules}(2))である。
    以上より$D_{\QCoh}(\cO_{D \times D})$の完全三角形
    \begin{equation}\label{eq:triangle-on-DD}
        \Delta_*\cO_D(-D)[1] \to \pi_{DD}(\pi_{DX}^*\cO_{\Gamma} \otimes \pi_{XD}\cO_{\overline{\Gamma}}) \to \Delta_*\cO_D \xrightarrow{+1}
    \end{equation}
    を得る。これらは例\ref{ex:line-bundle-as-Fourier-Mukai},\ref{ex:composition-of-Fourier-Mukai-over-field},\ref{ex:push-pull-as-Fourier-Mukai}より(ここで分離性と平坦性を使う)それぞれ$(-)\otimes \cO_D(-D)[1], i^*i_*, \id$のFourier--向井核だから、命題の完全三角形を得る。
    完全三角形の中の射の説明についても三角形\eqref{eq:triangle-on-DDD}の構成からわかる。
\end{proof}


\printbibliography[title=参考文献]
\end{document}