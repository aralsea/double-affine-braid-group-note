\RequirePackage{plautopatch}
\documentclass[uplatex, a4paper, dvipdfmx]{jsarticle}
\usepackage[backend=biber, style=alphabetic, sorting=nyt]{biblatex}
\addbibresource{../bib/double-affine-braid-group.bib} 

\usepackage{amsthm}
\usepackage{amsmath,amsfonts,amssymb}
\usepackage{tikz-cd}
\usepackage{dynkin-diagrams}

\renewcommand{\labelenumi}{(\arabic{enumi})}

\theoremstyle{definition}
\newtheorem{theorem}{定理}[section]
\newtheorem{definition}[theorem]{定義}
\newtheorem{proposition}[theorem]{命題}
\newtheorem{lemma}[theorem]{補題}
\newtheorem{corollary}[theorem]{系}
\newtheorem{fact}[theorem]{事実}
\newtheorem{example}[theorem]{例}
\newtheorem{remark}[theorem]{補足}

\DeclareMathOperator{\Hom}{\mathrm{Hom}}
\DeclareMathOperator{\Tor}{\mathrm{Tor}}
\DeclareMathOperator{\CHom}{\mathcal{H}\!\mathit{om}}
\DeclareMathOperator{\CTor}{\mathcal{T}\!\mathit{or}}
\DeclareMathOperator{\Auteq}{\mathrm{Auteq}}
\DeclareMathOperator{\Cone}{\mathrm{Cone}}
\DeclareMathOperator{\ev}{\mathrm{ev}}
\DeclareMathOperator{\id}{\mathrm{id}}
\DeclareMathOperator{\depth}{\mathrm{depth}}
\DeclareMathOperator{\Pic}{\mathrm{Pic}}
\DeclareMathOperator{\MCG}{\mathrm{MCG}}
\DeclareMathOperator{\PMCG}{\mathrm{PMCG}}
\DeclareMathOperator{\RHom}{\mathrm{RHom}}
\DeclareMathOperator{\Ker}{\mathrm{Ker}}
\DeclareMathOperator{\Image}{\mathrm{Im}}
\DeclareMathOperator{\Aut}{\mathrm{Aut}}
\DeclareMathOperator{\Inn}{\mathrm{Inn}}
\DeclareMathOperator{\Out}{\mathrm{Out}}
\DeclareMathOperator{\Supp}{\mathrm{Supp}}
\DeclareMathOperator{\SL}{\mathrm{SL}}
\DeclareMathOperator{\GL}{\mathrm{GL}}
\DeclareMathOperator{\Spec}{\mathrm{Spec}}
\DeclareMathOperator{\Proj}{\mathrm{Proj}}
\DeclareMathOperator{\Perf}{\mathrm{Perf}}
\DeclareMathOperator{\NS}{\mathrm{NS}}
\DeclareMathOperator{\Ext}{\mathrm{Ext}}
\DeclareMathOperator{\Hilb}{\mathrm{Hilb}}
\DeclareMathOperator{\res}{\mathrm{res}}
\DeclareMathOperator{\Ch}{\mathrm{Ch}}
\DeclareMathOperator{\Coh}{\mathrm{Coh}}
\DeclareMathOperator{\QCoh}{\mathrm{QCoh}}
\DeclareMathOperator{\Ann}{\mathrm{Ann}}
\DeclareMathOperator{\Aff}{\mathrm{Aff}}
\DeclareMathOperator{\FM}{\mathrm{FM}}
\DeclareMathOperator{\red}{\mathrm{red}}
\DeclareMathOperator{\length}{\mathrm{length}}
\DeclareMathOperator{\Mod}{\mathrm{Mod}}
\DeclareMathOperator{\tr}{\mathrm{tr}}



\newcommand{\nc}{\newcommand}

%% Calligraphic letters
\nc{\cD}{{\mathcal{D}}}
\nc{\cF}{{\mathcal{F}}}
\nc{\cG}{{\mathcal{G}}}
\nc{\cH}{{\mathcal{H}}}
\nc{\cJ}{{\mathcal{J}}}
\nc{\cM}{{\mathcal{M}}}
\nc{\cO}{{\mathcal{O}}}
\nc{\cU}{{\mathcal{U}}}
\nc{\cW}{{\mathcal{W}}}

%% Blackboard letters
\nc{\bA}{{\mathbb{A}}}
\nc{\bC}{{\mathbb{C}}}
\nc{\bP}{{\mathbb{P}}}
\nc{\bQ}{{\mathbb{Q}}}
\nc{\bR}{{\mathbb{R}}}
\nc{\bZ}{{\mathbb{Z}}}


%% Fraktur letters
\nc{\fg}{{\mathfrak{g}}}
\nc{\fu}{{\mathfrak{u}}}

% hyperref
\usepackage[urlcolor=blue]{hyperref}


\title{半捻り関手の一般論}
\author{荒井 勇人}
\date{\today}
\begin{document}
\maketitle
\section{記法}
\begin{itemize}
    \item 導来関手の$R$と$L$は原則省略する。
    \item $X$が環付き空間のとき$\Mod(\cO_X)$を$\cO_X$加群の圏、$D(\cO_X)$を$\Mod(\cO_X)$の導来圏とする。$+, -, b$をつけて有界導来圏を表す。
    \item $X$がスキームのとき、$D_{\QCoh}(\cO_X) \subset D(\cO_X)$をコホモロジーが凖連接層であるような複体のなす充満部分圏とする。
    \item $X$がNoetherスキームのとき、$D_{\Coh}(\cO_X) \subset D(\cO_X)$をコホモロジーが連接層であるような複体のなす充満部分圏とする。また$D^b(X) = D_{\Coh}^b(\cO_X)$とする。($X$がNoetherなのでこれは連接層の圏$\Coh(X)$の導来圏と同値である。)
\end{itemize}
\section{導来関手}
\begin{definition}
    $X$を環付き空間とし、$E \in D(\cO_X)$とする。このとき$E$の双対$E^\vee$を
    \begin{equation}
        E^\vee = \RHom(E, \cO_X)
    \end{equation}
    により定義する。
\end{definition}
\begin{proposition}
    $f \colon X \to Y$を環付き空間の射とし、$K, M \in D(\cO_Y)$とする。このとき自然な射
    \begin{equation}
        f^*\CHom(K, M) \to \CHom(f^*K, f^*M)
    \end{equation}
    は以下のいずれかの状況で同型である。
    \begin{enumerate}
        \item $K$がperfect
        \item $X, Y$がNoetherスキーム、$f$が平坦で、$K \in D_{\Coh}^-(\cO_Y), M \in D^+(\cO_Y)$
    \end{enumerate}
\end{proposition}
\begin{proof}
    {\cite[\href{https://stacks.math.columbia.edu/tag/0GM7}{Tag 0GM7}]{stacks-project}}より直ちに従う。
\end{proof}
\begin{corollary}\label{cor:pullback-and-dual}
    $f \colon X \to Y$をNoetherスキームの間の平坦射とすると、任意の$E \in D_{\Coh}^-(\cO_Y)$について双対は引き戻しと交換する。つまり$f^*(E^\vee) \cong (f^*E)^\vee$である。
\end{corollary}
\begin{proposition}\label{prop:tensor-and-dual}
    $X$を環付き空間とし、$L, E \in D(\cO_X)$とする。さらに$L$がperfectであるとする。このとき自然な同型
    \begin{equation}
        E^\vee \otimes L^\vee \cong (E \otimes L)^\vee
    \end{equation}
    が存在する。
\end{proposition}
\begin{proof}
    {\cite[\href{https://stacks.math.columbia.edu/tag/0G40}{Tag 0G40}]{stacks-project}}より自然な同型
    \begin{equation}
        \CHom(E, \cO_X) \otimes L^\vee \cong \CHom(\CHom(L^\vee, E), \cO_X)
    \end{equation}
    がある。
    特に$L^{\vee\vee} \cong L$もわかる。
    そして {\cite[\href{https://stacks.math.columbia.edu/tag/08DQ}{Tag 08DQ}]{stacks-project}}より
    \begin{equation}
        \CHom(L^\vee, E) \cong E \otimes L^{\vee\vee} \cong E \otimes L
    \end{equation}
    なので命題が成り立つ。
\end{proof}
\section{Grothendieck--Verdier双対}
\begin{theorem}[{\cite[\href{https://stacks.math.columbia.edu/tag/0A9E}{Tag 0A9E}]{stacks-project}}]
    $f \colon X \to Y$をqcqsスキームの間の射とする。このとき
    \begin{equation}
        f_* \colon D_{\QCoh}(\cO_X) \to D_{\QCoh}(\cO_Y)
    \end{equation}
    は右随伴$a \colon D_{\QCoh}(\cO_Y) \to D_{\QCoh}(\cO_X)$を持つ。さらにこれは$D_{\QCoh}^+$を保つ。
\end{theorem}
\begin{theorem}[{\cite[\href{https://stacks.math.columbia.edu/tag/0GEW}{Tag 0GEW}]{stacks-project}}]\label{thm:Grothendieck-Verdier-duality-for-complexes}
    $f \colon X \to Y$をNoetherスキームの間の固有射とする。このとき任意の$L \in D_{\Coh}^-(\cO_X), K \in D_{\QCoh}^+(\cO_Y)$に対して自然な同型
    \begin{equation}
        f_* \CHom(L, a(K)) \xrightarrow{\sim} \CHom(f_*L, K)
    \end{equation}
    がある。
\end{theorem}
\begin{definition}[{\cite[\href{https://stacks.math.columbia.edu/tag/0A74}{Tag 0A74}]{stacks-project}}]
    $i \colon Z \to X$をスキームの閉埋め込みとする。$\cO_X$加群$F$に対して、層$\CHom_{\cO_X}(i_*\cO_Z, F)$は自然に$\cO_Z$加群とみなせる。
    これにより定まる関手$\Mod(\cO_X) \to \Mod(\cO_Z)$を$\CHom(\cO_Z, -)$と書く。
\end{definition}
\begin{proposition}[{\cite[\href{https://stacks.math.columbia.edu/tag/0A74}{Tag 0A74}]{stacks-project}}]\label{prop:Grothendieck-Verdier-duality-for-closed-immersion}
    $i \colon Z \to X$をスキームの閉埋め込みとする。このとき
    \begin{enumerate}
        \item 関手$\CHom(\cO_Z, -) \colon \Mod(\cO_X) \to \Mod(\cO_Z)$は左完全である。
        \item その右導来関手を$R\CHom(\cO_Z, -) \colon D(\cO_X) \to D(\cO_Z)$とおくと、これは$(L i_* = )i_* = R i_* \colon D(\cO_Z) \to D(\cO_X)$の右随伴である。
        \item $R\CHom(\cO_Z, -)$は$D_{\QCoh}^+$を保つ。
        \item $i_* \colon D_{\QCoh}(\cO_Z) \to D_{\QCoh}(\cO_X)$の右随伴$a$と$R\CHom(\cO_Z, -)$は$D_{\QCoh}^+$へ制限したときに一致する。
    \end{enumerate}
\end{proposition}
\section{射影公式}
\begin{proposition}[射影公式]\label{prop:projection-formula-for-modules}
    $f \colon X \to Y$を環付き空間の射とし、$E \in D(\cO_X), K \in D(\cO_Y)$とする。このとき自然な射
    \begin{equation}
        f_*E \otimes K \to f_*(E \otimes f^*K)
    \end{equation}
    は以下の状況で同型である。
    \begin{enumerate}
        \item $K$がperfect({\cite[\href{https://stacks.math.columbia.edu/tag/0B54}{Tag 0B54}]{stacks-project}})
        \item $f$が閉集合への同相({\cite[\href{https://stacks.math.columbia.edu/tag/0B55}{Tag 0B55}]{stacks-project}})
    \end{enumerate}
\end{proposition}
\section{Fourier--Mukai変換}
$X, Y$を$S$上のスキームとし、$K \in D(\cO_{X \times_S Y})$とする。
\begin{equation}
    \Phi_K = p_{Y*}(p_{X,*}(-)\otimes_{\cO_{X \times_S Y}} K) \colon D(\cO_X) \to D(\cO_Y)
\end{equation}
を$K$に付随するFourier--向井関手という。$D_{\QCoh}$などの部分圏の間の関手になっているとき、それもFourier--向井関手という。

\begin{lemma}[{\cite[\href{https://stacks.math.columbia.edu/tag/0FYR}{Tag 0FYR}]{stacks-project}}]
    $X \to S$がqcqs射で$K \in D_{\QCoh}$なら$\Phi_K \colon D_{\QCoh}(\cO_X) \to D_{\QCoh}(\cO_Y)$である。
\end{lemma}
\begin{lemma}[{\cite[\href{https://stacks.math.columbia.edu/tag/0FYS}{Tag 0FYS}]{stacks-project}}]
    $X, Y, Z$が$S$上qcqsなスキームで、$X$と$Z$が$S$上tor independentかつ$Y$が$S$上平坦だとする。$K, K' \in D_{\QCoh}$のとき
    \begin{equation}
        K'' = p_{XZ,*}(p_{XY}^*K \otimes p_{YZ}^*K'')
    \end{equation}
    とおくと、
    \begin{equation}
        \Phi_{K'} \circ \Phi_K = \Phi_{K''}\colon D_{\QCoh}(\cO_X) \to D_{\QCoh}(\cO_Z)
    \end{equation}
    である。
\end{lemma}
\begin{example}\label{ex:composition-of-Fourier-Mukai-over-field}
    $S$が体$k$で$X, Y, Z$が$k$上qcqsなら使える。
\end{example}
\begin{example}\label{ex:push-pull-as-Fourier-Mukai}
    $f \colon X \to Y$を$S$上分離的なスキームの間の射とする。このとき以下で定まる$f$のグラフ$\Gamma_f \colon X \to X \times_S Y$は閉埋め込みである。
    \[
        \begin{tikzcd}
            X \ar[rd, "\Gamma_f"] \ar[rdd, bend right, "\id_X"] \ar[rrd, bend left, "f"]& & \\
            & X \times_S Y \ar[r] \ar[d] & Y \ar[d]\\
            & X            \ar[r]        & S
        \end{tikzcd}
    \]
    そして
    \begin{align}
        \Phi_{\cO_{\Gamma_f}}^{X \to Y} = f_* & \colon D(\cO_X) \to D(\cO_Y) \\
        \Phi_{\cO_{\Gamma_f}}^{Y \to X} = f_* & \colon D(\cO_Y) \to D(\cO_X)
    \end{align}
    である。
\end{example}
\begin{proof}
    $\Gamma_f$は閉埋め込みなので、\ref{prop:projection-formula-for-modules}より射影公式が使える。
    よって
    \begin{align}
        \Phi_{\cO_{\Gamma_f}}^{X \to Y}(-) & = p_{Y*}(p_X^*(-) \otimes \cO_{\Gamma_f})               \\
                                           & = p_{Y*}(p_X^*(-) \otimes \Gamma_{f*}\cO_X)             \\
                                           & = p_{Y*}\Gamma_{f*}(\Gamma_{f}^*p_X^*(-) \otimes \cO_X) \\
                                           & = f_*(-)
    \end{align}
    \begin{align}
        \Phi_{\cO_{\Gamma_f}}^{X \to Y}(-) & = p_{X*}(p_Y^*(-) \otimes \cO_{\Gamma_f})               \\
                                           & = p_{X*}(p_Y^*(-) \otimes \Gamma_{f*}\cO_X)             \\
                                           & = p_{X*}\Gamma_{f*}(\Gamma_{f}^*p_Y^*(-) \otimes \cO_X) \\
                                           & = f^*(-)
    \end{align}
    となる。
\end{proof}
\begin{example}\label{ex:line-bundle-as-Fourier-Mukai}
    $X$を$S$上分離的なスキームとし、$L$を$X$上の可逆層とする。このとき
    \begin{equation}
        \Phi_{\Delta_*L}(-) = L \otimes f^*(-) \colon D(\cO_X) \to D(\cO_Y)
    \end{equation}
    である。
\end{example}
\begin{proof}
    $\Delta$は閉埋め込みなので、\ref{prop:projection-formula-for-modules}より射影公式が使える。
    よって
    \begin{align}
        \Phi_{\Delta_*L}(-) & = p_{Y*}(p_X^*(-) \otimes \Delta_* L)        \\
                            & = p_{Y*}\Delta_*(\Delta^*p_X^*(-) \otimes L) \\
                            & = (-) \otimes L
    \end{align}
    である。
\end{proof}

\section{Cartier因子}
\begin{definition}
    $X$をスキームとする。閉部分スキーム$D \subset X$が有効Cartier因子であるとは、$D$の定義イデアル$I$が可逆層であることをいう。
    このとき$\cO_X(-D) = I, \cO_X(D) = \cO_X(-D)^{-1}$と定める。
    定義から完全列
    \begin{equation}
        0 \to \cO_X(-D) \to \cO_X \to \cO_D \to 0
    \end{equation}
    が存在する。

\end{definition}
\begin{proposition}
    $X$をスキームとし、$D \subset X$を閉部分スキームとする。このとき以下は同値である。
    \begin{enumerate}
        \item $D$は有効Cartier因子である。
        \item $D$の定義イデアルは局所的に非零因子で生成される。
    \end{enumerate}
\end{proposition}
\begin{proposition}\label{prop:duality-for-Cartier-divisor}
    $X$をスキームとし、$i \colon D\hookrightarrow X$を有効Cartier因子とする。
    このとき関手
    \begin{equation}
        i^*(-)\otimes \cO_D(D)[-1] \colon D(\cO_X) \to D(\cO_D)
    \end{equation}
    は関手$i_* \colon D(\cO_D) \to D(\cO_X)$の右随伴である。
    さらにこれは$D_{\QCoh}$を保ち、
    特に$i_* \colon D_{\QCoh}(\cO_D) \to D_{\QCoh}(\cO_X)$の右随伴$a$と一致する。
\end{proposition}
\begin{proof}
    まず{\cite[\href{https://stacks.math.columbia.edu/tag/0AA4}{Tag 0AA4}]{stacks-project}}より$i^*(-)\otimes \cO_D(D)[-1] \cong R\CHom(\cO_D, -)$だから、命題\ref{prop:Grothendieck-Verdier-duality-for-closed-immersion}より$i^*(-)\otimes \cO_D(D)[-1]$は$i_*$の右随伴である。
    さらに$i^*(-)\otimes \cO_D(D)[-1]$は明らかに$D_{\QCoh}$を保つ。
    よって誘導される関手$i^*(-)\otimes \cO_D(D)[-1] \colon D_{\QCoh}(\cO_X) \to D_{\QCoh}(\cO_D)$も$i_*$の右随伴であり、特に$a$と一致する。
\end{proof}

\begin{proposition}\label{prop:triangle-for-push-pull}
    $k$を体とし、$X$を$k$上分離的なNoetherスキーム、$i \colon D \hookrightarrow X$を有効Cartier因子とする。
    このとき任意の$F \in D_{\QCoh}(\cO_D)$に対して、$D_{\QCoh}(\cO_Y)$の完全三角形
    \begin{equation}
        F \otimes \cO_D(-D)[1] \to i^*i_*F \to F \xrightarrow{+1}
    \end{equation}
    が存在する。ここで$i^*i_*F \to F$は随伴$i^* \dashv i_*$の余単位射、$F \otimes \cO_D(-D)[1] \to i^*i_*F$は
    \begin{equation}
        \Hom(F \otimes \cO_D(-D)[1], i^*i_*F) \cong \Hom(F, i^*i_*F\otimes \cO_D(D)[-1]) = \Hom(F, a(i_*F))
    \end{equation}
    によって随伴$i_* \dashv a$の単位射に対応するものである。

    $F = \cO_D$のときはより強く、
    \begin{equation}
        i^*i_*\cO_D \cong \cO_D \oplus \cO_D(-D)[-1]
    \end{equation}
    が成り立つ。
\end{proposition}
\begin{proof}
    \cite[corollary 11.4]{MR2244106}の証明がこの場合に使えることを確かめる。

    まず$F = \cO_D$のときに示す。
    完全列
    \begin{equation}\label{eq:koszul-resolution}
        0 \to \cO_X(-D) \to \cO_X \to i_*\cO_D \to 0
    \end{equation}
    が$i_*\cO_D$の平坦分解を与えるから、これにより$i^*$を計算すればよい。

    次に$F$が(一般の複体ではなく)準連接層のときに成立することを示す。
    射影公式よりと分解\eqref{eq:koszul-resolution}より
    \begin{align}
        i_*i^*i_*F & \cong i_*\cO_D \otimes i_*F                                     \\
                   & \cong (0 \to i_*F\otimes\cO_X(-D) \xrightarrow{d} i_*F \to 0)   \\
                   & \cong (0 \to i_*(F\otimes\cO_D(-D)) \xrightarrow{d} i_*F \to 0)
    \end{align}
    となり、微分$d$は$D$に制限されているので$0$となる。よって$i^*i_*F$のコホモロジーは($i_*$してから計算できるので)
    \[
        \cH^j(i^*i_*F) \cong \begin{cases}
            F                   & (j = 0)            \\
            F \otimes \cO_D(-D) & (j = -1)           \\
            0                   & (\text{otherwise})
        \end{cases}
    \]
    となる。
    特に$i^*i_*F$は2項複体だから、完全三角形
    \begin{equation}
        \cH^{-1}[1] \to i^*i_*F \to \cH^0 \xrightarrow{+1}
    \end{equation}
    が存在する。
    あとは
    右端のコホモロジーへの自然な射$i^*i_*F \to \cH^0 \cong F$が随伴$i^*\dashv i_*$の余単位射に、
    左端のコホモロジーからの自然な射$F \otimes \cO_D(-D)[1] \cong \cH^{-1}[1] \to i^*i_*F$が随伴$i_* \dashv a$の単位射に
    対応することを示せばよい。
    $P^\bullet \to i_*F$と$i_*F \to I^\bullet$をそれぞれ$i_*F$の平坦分解と単射分解とする。
    このとき$\id_{i_*F}$は同型
    \begin{equation}
        \Hom(i_*F, i_*F) \cong \Hom(P^\bullet, i_*F) \cong \Hom(i^*P^\bullet, F)
    \end{equation}
    を通して$P^\bullet \to i_*F$の随伴
    \[
        \begin{tikzcd}
            \cdots \ar[r] & i^*P^{-1} \ar[r] \ar[d] & i^*P^0 \ar[r] \ar[d] & 0 \ar[d]\ar[r]& \cdots \\
            \cdots \ar[r] & 0 \ar[r]     & F \ar[r]     & 0\ar[r]& \cdots
        \end{tikzcd}
    \]
    に対応する。これは右端のコホモロジーへの自然な射に他ならない。
    同様に同型
    \begin{equation}
        \Hom(i_*F, i_*F) \cong \Hom(i_*F, I^\bullet) \cong \Hom(F, a(I^\bullet)) \cong \Hom(F, R\CHom(\cO_D, I^\bullet))
    \end{equation}
    により$\id_{i_*F}$は
    \[
        \begin{tikzcd}
            \cdots \ar[r] & 0 \ar[r]  \ar[d]    & F \ar[r]\ar[d]     & 0\ar[r]\ar[d]& \cdots\\
            \cdots \ar[r] & 0 \ar[r]  & R\CHom(\cO_D, I^0) \ar[r] & R\CHom(\cO_D, I^1)\ar[r]& \cdots
        \end{tikzcd}
    \]
    に対応し、これは左端のコホモロジーからの自然な射に他ならない。

    最後一般の$F \in D_{\QCoh}(\cO_D)$に対して示す。
    $i \colon D \hookrightarrow X$のグラフを$\Gamma \subset D \times X$、その転置を$\overline{\Gamma} \subset X \times D$とする。
    \[
        \begin{tikzcd}
            && D &\\
            &\Gamma \times D \ar[r, hook, "\varphi"]\ar[d, "\pi"] &D \times D \times D \ar[d, hook, "\psi"] \ar[u, "\pi_2"] & D \times \overline{\Gamma} \ar[l, hook', "\overline{\varphi}"]\\
            D  \ar[r, hook, "\Delta"] \ar[ru, bend left, "\eta"]&D \times D & D \times X \times D \ar[l, "\pi_{DD}"] &
        \end{tikzcd}
    \]
    という可換図式を考える。ただし$\eta$は$D \xrightarrow{\sim} \Gamma$と$\id_D$の積である。
    有効Cartier因子$\varphi \colon D \times D \times D \hookrightarrow D \times X \times D$と$D \times D \times D$上の準連接層$\overline{\varphi}_*\cO_{D \times \overline{\Gamma}}$に対して命題を適用すると、完全三角形
    \begin{equation}\label{eq:triangle-on-DDD}
        \overline{\varphi}_* \cO_{D \times \overline{\Gamma}}\otimes \pi_2^*\cO_D(-D)[1]\to \psi^*\psi_*\overline{\varphi}_*\cO_{D \times \overline{\Gamma}}\to \overline{\varphi}_*\cO_{D \times \overline{\Gamma}} \xrightarrow{+1}
    \end{equation}
    を得る。
    $\eta \colon D \hookrightarrow D \times D \times D$を対角射とすると、
    \begin{equation}
        \varphi^*\overline{\varphi}_*\cO_{X \times \overline{\Gamma}} \cong \eta_*\cO_D
    \end{equation}
    かつ
    \begin{equation}
        \varphi^*(\overline{\varphi}_* \cO_{D \times \overline{\Gamma}}\otimes \pi_2^*\cO_D(-D)[1]) \cong \eta_*\cO_D(-D)[1]
    \end{equation}
    だから、\eqref{eq:triangle-on-DDD}を$\varphi$で引き戻すことで$D_{\QCoh}(\cO_{\Gamma \times D})$の完全三角形
    \begin{equation}
        \eta_*\cO_D(-D)[1] \to \varphi^*\psi^*\psi_*\overline{\varphi}_*\cO_{D \times \overline{\Gamma}} \to \eta_*\cO_D \xrightarrow{+1}
    \end{equation}
    を得る。
    これをさらに$\pi$で押し出すと
    \begin{equation}
        \Delta_*\cO_D(-D)[1] \to \pi_*\varphi^*\psi^*\psi_*\overline{\varphi}_*\cO_{D \times \overline{\Gamma}} \to \Delta_*\cO_D \xrightarrow{+1}
    \end{equation}
    という$D_{\QCoh}(\cO_{D \times D})$の完全三角形となる。
    ここで
    \begin{align}
        \pi_*\varphi^*\psi^*\psi_*\overline{\varphi}_*\cO_{D \times \overline{\Gamma}} & \cong \pi_{DD}((\psi \circ \varphi)_*\cO_{\Gamma \times D} \otimes (\psi \circ \overline{\varphi})_*\cO_{D \times \overline{\Gamma}}) \\
                                                                                       & \cong \pi_{DD}(\pi_{DX}^*\cO_{\Gamma} \otimes \pi_{XD}\cO_{\overline{\Gamma}})
    \end{align}
    が成り立つ。1つ目の同型は$\psi \circ \varphi$についての射影公式(命題\ref{prop:projection-formula-for-modules}(2))である。
    以上より$D_{\QCoh}(\cO_{D \times D})$の完全三角形
    \begin{equation}\label{eq:triangle-on-DD}
        \Delta_*\cO_D(-D)[1] \to \pi_{DD}(\pi_{DX}^*\cO_{\Gamma} \otimes \pi_{XD}\cO_{\overline{\Gamma}}) \to \Delta_*\cO_D \xrightarrow{+1}
    \end{equation}
    を得る。これらは例\ref{ex:line-bundle-as-Fourier-Mukai},\ref{ex:composition-of-Fourier-Mukai-over-field},\ref{ex:push-pull-as-Fourier-Mukai}より(ここで体上であることと分離性を使う)それぞれ$(-)\otimes \cO_D(-D)[1], i^*i_*, \id$のFourier--向井核だから、命題の完全三角形を得る。
    完全三角形の中の射の説明についても三角形\eqref{eq:triangle-on-DDD}の構成からわかる。
\end{proof}

\section{共役作用}
この節では$X \to T$を体$k$上の非特異多様体の間の平坦射とし、$\dim T = 1$とする。また$i \colon Y \hookrightarrow X$を$X \to T$の$0 \in T$でのファイバーとする。
さらに以下のように射を定める。
\[
    \begin{tikzcd}
        Y \arrow[r,"i"]\arrow[d, "\Delta_Y"] & X \arrow[d,"d"]\arrow[rdd, bend left, "\Delta_X"]&\\
        Y \times Y \arrow[r, "k"]\arrow[rrd, bend right, "j=i \times i"]& X \times_T X\arrow[rd, "\iota"]&\\
        & & X \times X.
    \end{tikzcd}
\]
$i_* \colon D_{\QCoh}(\cO_Y) \to D_{\QCoh}(\cO_X)$の右随伴を$a$とおく。
\begin{lemma}
    $k \colon Y \times Y \hookrightarrow X \times_T X$は有効Cartier因子である。
\end{lemma}
\begin{proof}
    以下のファイバー積の図式がある。
    \[
        \begin{tikzcd}
            Y \times Y \arrow[r, "k"]\arrow[d] & X \times_T X \arrow[d]\\
            0 \arrow[r]& T
        \end{tikzcd}
    \]
    $T$は非特異曲線だから、$0 \to T$は有効Cartier因子である。
    有効Cartier因子は平坦射で引き戻せる(  {\cite[\href{https://stacks.math.columbia.edu/tag/02OO}{Tag 02OO}]{stacks-project}})から$k$も有効Cartier因子である。
\end{proof}
\begin{lemma}
    $\cO_X(-Y) \cong \cO_X, \cO_{X \times_T X}(-Y\times Y) \cong \cO_{X \times_T X}$である。
\end{lemma}
\begin{proof}
    $0 \to T$のイデアルが自明束であることから従う。
\end{proof}
\begin{proposition}
    $E \in D(\cO_Y)$とし、$E' = \CHom(E, a(\cO_Y))$とおく。このとき$E'\cong \CHom(E, \cO_Y)[-1] = E^\vee[-1]$である。さらに$E \in D_{\Coh}^-(\cO_Y)$ならば$i_*E' \cong (i_*E)^\vee$である。
\end{proposition}
\begin{proof}
    命題\ref{prop:duality-for-Cartier-divisor}より$a(\cO_Y) \cong \cO_Y(-Y)[-1]$である。$\cO_Y(-Y) \cong \cO_Y$だから、$\CHom(E, a(\cO_Y)) \cong \CHom(E, \cO_Y)[-1]$である。
    また定理\ref{thm:Grothendieck-Verdier-duality-for-complexes}より同型
    \begin{equation}
        i_*(E') = i_* \CHom(E, a(\cO_Y)) \cong \CHom(i_*E, \cO_X) = (i_*E)^\vee
    \end{equation}
    がある。
\end{proof}
$E \in D_{\QCoh}(\cO_Y)$とし、$E' \boxtimes E \cong E^\vee \boxtimes E[-1] \in D_{\QCoh}(\cO_{Y\times Y})$と$k \colon Y \times Y \hookrightarrow X \times_T X$に命題\ref{prop:triangle-for-push-pull}を適用すると
\begin{equation}\label{eq:triangle-for-E-boxtimes-E}
    (E' \boxtimes E) \otimes \cO_{Y \times Y}(-Y\times Y)[1] \to k^*k_*(E' \boxtimes E) \to E' \boxtimes E \xrightarrow{+1}
\end{equation}
という完全三角形を得る。この境界射を$\delta \colon E' \boxtimes E[-1] \to (E' \boxtimes E) \otimes \cO_{Y \times Y}(-Y\times Y)[1]$とおく。
\begin{proposition}
    $\delta$と$(E' \boxtimes E) \otimes \cO_{Y \times Y}(-Y\times Y)[1] \cong E^\vee \boxtimes E \xrightarrow{\tr} \cO_{\Delta_Y} = \Delta_{Y*}\cO_Y$の合成$\tr \circ \delta$は$0$である。
\end{proposition}
\begin{proof}
    完全三角形\eqref{eq:triangle-for-E-boxtimes-E}にコホモロジー的関手$\Hom(-, \cO_{\Delta_Y})$を適用すると、完全列
    \begin{equation}
        \cdots \to \Hom(k^*k_*(E' \boxtimes E), \cO_{\Delta_Y}) \to \Hom((E' \boxtimes E) \otimes \cO_{Y \times Y}(-Y\times Y)[1], \cO_{\Delta_Y}) \xrightarrow{-\circ \delta} \Hom(E' \boxtimes E[-1], \cO_{\Delta_Y}) \to \cdots
    \end{equation}
    を得る。よって$\tr \circ \delta = 0$を示すには、$\tr$が$k^*k_*(E' \boxtimes E)$を経由することを示せば十分である(実際には同値)。
    ここで
    \begin{enumerate}
        \item $\tr \colon(E' \boxtimes E) \otimes \cO_{Y \times Y}(-Y\times Y)[1] \to \cO_{\Delta_Y}$が$k^*k_*(E' \boxtimes E)$を経由する。
        \item $E'\boxtimes E \to \cO_{\Delta_Y} \otimes  \cO_{Y \times Y}(Y\times Y)[-1] \cong a(d_*\cO_X)$が$k^*k_*(E' \boxtimes E) \otimes  \cO_{Y \times Y}(Y\times Y)[-1] \cong a(k_*(E' \boxtimes E))$を経由する。
    \end{enumerate}
    と同値変形する。$\cO_{\Delta_Y} = \Delta_{Y*}\cO_Y \cong \Delta_{Y*}i^*\cO_X \cong k^*d_*\cO_X$(最後の同型はtor-independent base change)から$\cO_{\Delta_Y} \otimes \cO_{Y \times Y}(Y\times Y)[-1] \cong a(d_*\cO_X)$が成り立つことに注意する。
    (2)は図式
    \[
        \begin{tikzcd}
            E' \boxtimes E \ar[r, "\eta"] \ar[rd, bend right] &a(k_*(E' \boxtimes E))\ar[d, dashed] \\
            & a(d_*\cO_X)
        \end{tikzcd}
    \]
    を可換にするような点線の射が存在するかという命題だが、$\eta$は随伴$k_* \dashv a$の単位射だったから、随伴で移すと
    \[
        \begin{tikzcd}
            k_*(E' \boxtimes E) \ar[r, equal] \ar[rd, bend right] &k_*(E' \boxtimes E)\ar[d, dashed] \\
            & d_*\cO_X
        \end{tikzcd}
    \]
    という点線の射を作ることに帰着できる。これは自明である。
\end{proof}

% ここで{\cite{MR4452435}}の議論を引用する。$\cD$を三角圏とし、$A \xrightarrow{f} B \xrightarrow{g} C$を$\cD$の複体、つまり$g \circ f=0$を満たす射の組とする。
% この複体に付随する左Postnikov系とは、可換図式
% \[
%     \begin{tikzcd}
%         A \ar[r, "f"] & B \ar[r, "g"]\ar[d] & C \\
%         & \Cone(f)\ar[lu, dashed, bend left, "+1"] \ar[ru, bend right, "m"] &
%     \end{tikzcd}
% \]
% のことである。$m$の取り方は一意的ではないため、左Postnikov系は一意ではない。$\Cone(m)$をこの左Postnikov系の畳み込みという。
% 同様に右Postnikov系とその畳み込みを図式
% \[
%     \begin{tikzcd}
%         A \ar[r, "f"] \ar[rd, bend right, dashed, "j"]& B \ar[r, "g"] & C \ar[ld, bend left, "h"] \\
%         & \Cone(g)\ar[u, dashed, "+1"] &
%     \end{tikzcd}
% \]
% を使って定義する。

% $\cD$の複体$A \to B \to C$にたいし、その左もしくは右Postnikov系の畳み込みになっているような任意の対象$E \in \cD$を、その複体の畳み込みと呼ぶ。
% \begin{lemma}[{\cite{MR4452435}}]
%     $\Ext^{-1}_{\cD}(A, C) = 0$ならば、複体$A \to B \to C$の畳み込みは同型をのぞいて一意に存在する。
% \end{lemma}


% \begin{proposition}
%     $E \in D^b(Y)$を$i \colon Y \hookrightarrow X$についての半捻り対象とし、対応する半捻り関手を$H_E \in \Auteq(D^b(Y))$とする。
%     このとき$H_E$のFourier--向井核は$D_{\QCoh}(\cO_{Y \times Y})$の複体
%     \begin{equation}
%         E^\vee \boxtimes E[-2] \xrightarrow{\delta} E^\vee \boxtimes E \xrightarrow{\tr} \cO_{\Delta_Y}
%     \end{equation}
%     の(一意な)畳み込みである。
% \end{proposition}
% \begin{proof}
%     $H_E$のFourier--向井核は
%     \begin{equation}
%         \Cone(k_*k^*(E' \boxtimes E) \to \cO_{\Delta_Y})
%     \end{equation}
%     だった。
%     % TODO
% \end{proof}

% \begin{proposition}[証明できていない!]
%     $E \in D^b(Y)$を$i$についての半球面対象とする。
%     さらに$L \in \Pic(Y)$を、$E \otimes L$も半球面対象となるような直線束とする。
%     このとき
%     \begin{equation}
%         (L \otimes -) \circ H_E \circ (L \otimes -)^{-1} \cong H_{E\otimes L}
%     \end{equation}
%     が成り立つ。
% \end{proposition}
% \begin{proof}
%     $\Delta \colon Y \to Y \times Y$を対角射とする。
%     $(L \otimes -) \circ H_E \cong H_{E\otimes L} \circ (L \otimes -)$を示す。
%     \begin{align}
%         P & = \pi_1^*E^\vee \otimes \pi_2^* E \otimes \pi_{23}^*\Delta_*L                         \\
%         Q & = \pi_{12}^*\Delta_*L \otimes \pi_2^*(E \otimes L)^\vee \otimes \pi_3^* (E \otimes L)
%     \end{align}
%     とおく。
%     これによりそれぞれのFourier--向井核は
%     \begin{align}
%         \Cone(\Cone(\pi_{13*}P[-2] \to \pi_{13*}P) \to \Delta_*L) \\
%         \Cone(\Cone(\pi_{13*}Q[-2] \to \pi_{13*}Q) \to \Delta_*L)
%     \end{align}
%     となる。

%     まず$\pi_{13*}P \cong \pi_{13*}Q$を示す。$l = \id \times \Delta \colon Y\times Y \to Y \times Y \times Y$とする。
%     \[
%         \begin{tikzcd}
%             Y & Y \times Y \ar[l, "p_1"]\ar[r, "l"] \ar[d, "p_2"]& Y \times Y \times Y \ar[d, "\pi_{23}"]\\
%             & Y \ar[r, "\Delta"] & Y \times Y
%         \end{tikzcd}
%     \]
%     上の図式は平坦射$\pi_{23}$によるbase changeだから、$\pi_{23}^*\Delta_*L \cong l_*p_2^*L$である。
%     また$p_1 = \pi_1 \circ l, p_2 = \pi_2 \circ l$であることに注意し、系\ref{cor:pullback-and-dual}や命題\ref{prop:tensor-and-dual}などを使うと
%     \begin{align}
%         P & =  \pi_1^*E^\vee \otimes \pi_2^* E \otimes \pi_{23}^*\Delta_*L \\
%           & \cong  \pi_1^*E^\vee \otimes \pi_2^* E \otimes l_*p_2^*L       \\
%           & \cong l_*(l^*(\pi_1^*E^\vee \otimes \pi_2^* E) \otimes p_2^*L) \\
%           & \cong l_*(p_1^*E^\vee \otimes p_2^*E \otimes p_2^*L)           \\
%           & \cong l_*(p_1^*E^\vee \otimes p_2^*(E \otimes L))
%     \end{align}
%     となる。同様に$m = \Delta \times \id \colon Y \times Y \to Y \times Y \times Y$とすると
%     \begin{align}
%         Q & = \pi_{12}^*\Delta_*L \otimes \pi_2^*(E \otimes L)^\vee \otimes \pi_3^* (E \otimes L)  \\
%           & \cong m_*p_1^*L \otimes \pi_2^*(E \otimes L)^\vee \otimes \pi_3^* (E \otimes L)        \\
%           & \cong m_*(p_1^*L \otimes m^*(\pi_2^*(E \otimes L)^\vee \otimes \pi_3^* (E \otimes L))) \\
%           & \cong m_*(p_1^*L \otimes p_1^*(E \otimes L)^\vee \otimes p_2^* (E \otimes L))          \\
%           & \cong m_*(p_1^*E^\vee \otimes p_2^* (E \otimes L))
%     \end{align}
%     となる。さらに$\pi_{13} \circ l = \pi_{13} \circ m = \id_{Y \times Y}$だから、
%     \begin{align}
%         \pi_{13*}P & \cong p_1^*E^\vee \otimes p_2^*(E \otimes L) \\
%                    & \cong \pi_{13*}Q
%     \end{align}
%     となる。
% \end{proof}

\printbibliography[title=参考文献]
\end{document}