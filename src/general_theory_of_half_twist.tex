\RequirePackage{plautopatch}
\documentclass[uplatex, a4paper, dvipdfmx]{jsarticle}
\usepackage[backend=biber, style=alphabetic, sorting=nyt]{biblatex}
\addbibresource{../bib/double-affine-braid-group.bib} 

\usepackage{amsthm}
\usepackage{amsmath,amsfonts,amssymb}
\usepackage{tikz-cd}
\usepackage{dynkin-diagrams}

\renewcommand{\labelenumi}{(\arabic{enumi})}

\theoremstyle{definition}
\newtheorem{theorem}{定理}[section]
\newtheorem{definition}[theorem]{定義}
\newtheorem{proposition}[theorem]{命題}
\newtheorem{lemma}[theorem]{補題}
\newtheorem{corollary}[theorem]{系}
\newtheorem{fact}[theorem]{事実}
\newtheorem{example}[theorem]{例}
\newtheorem{remark}[theorem]{補足}

\DeclareMathOperator{\Hom}{\mathrm{Hom}}
\DeclareMathOperator{\Tor}{\mathrm{Tor}}
\DeclareMathOperator{\CHom}{\mathcal{H}\!\mathit{om}}
\DeclareMathOperator{\CTor}{\mathcal{T}\!\mathit{or}}
\DeclareMathOperator{\Auteq}{\mathrm{Auteq}}
\DeclareMathOperator{\Cone}{\mathrm{Cone}}
\DeclareMathOperator{\ev}{\mathrm{ev}}
\DeclareMathOperator{\id}{\mathrm{id}}
\DeclareMathOperator{\depth}{\mathrm{depth}}
\DeclareMathOperator{\Pic}{\mathrm{Pic}}
\DeclareMathOperator{\MCG}{\mathrm{MCG}}
\DeclareMathOperator{\PMCG}{\mathrm{PMCG}}
\DeclareMathOperator{\RHom}{\mathrm{RHom}}
\DeclareMathOperator{\Ker}{\mathrm{Ker}}
\DeclareMathOperator{\Image}{\mathrm{Im}}
\DeclareMathOperator{\Aut}{\mathrm{Aut}}
\DeclareMathOperator{\Inn}{\mathrm{Inn}}
\DeclareMathOperator{\Out}{\mathrm{Out}}
\DeclareMathOperator{\Supp}{\mathrm{Supp}}
\DeclareMathOperator{\SL}{\mathrm{SL}}
\DeclareMathOperator{\GL}{\mathrm{GL}}
\DeclareMathOperator{\Spec}{\mathrm{Spec}}
\DeclareMathOperator{\Proj}{\mathrm{Proj}}
\DeclareMathOperator{\Perf}{\mathrm{Perf}}
\DeclareMathOperator{\NS}{\mathrm{NS}}
\DeclareMathOperator{\Ext}{\mathrm{Ext}}
\DeclareMathOperator{\Hilb}{\mathrm{Hilb}}
\DeclareMathOperator{\res}{\mathrm{res}}
\DeclareMathOperator{\Ch}{\mathrm{Ch}}
\DeclareMathOperator{\Coh}{\mathrm{Coh}}
\DeclareMathOperator{\QCoh}{\mathrm{QCoh}}
\DeclareMathOperator{\Ann}{\mathrm{Ann}}
\DeclareMathOperator{\Aff}{\mathrm{Aff}}
\DeclareMathOperator{\FM}{\mathrm{FM}}
\DeclareMathOperator{\red}{\mathrm{red}}
\DeclareMathOperator{\length}{\mathrm{length}}
\DeclareMathOperator{\Mod}{\mathrm{Mod}}



\newcommand{\nc}{\newcommand}

%% Calligraphic letters

\nc{\cF}{{\mathcal{F}}}
\nc{\cG}{{\mathcal{G}}}
\nc{\cH}{{\mathcal{H}}}
\nc{\cJ}{{\mathcal{J}}}
\nc{\cM}{{\mathcal{M}}}
\nc{\cO}{{\mathcal{O}}}
\nc{\cU}{{\mathcal{U}}}
\nc{\cW}{{\mathcal{W}}}

%% Blackboard letters
\nc{\bA}{{\mathbb{A}}}
\nc{\bC}{{\mathbb{C}}}
\nc{\bP}{{\mathbb{P}}}
\nc{\bQ}{{\mathbb{Q}}}
\nc{\bR}{{\mathbb{R}}}
\nc{\bZ}{{\mathbb{Z}}}


%% Fraktur letters
\nc{\fg}{{\mathfrak{g}}}
\nc{\fu}{{\mathfrak{u}}}

% hyperref
\usepackage[urlcolor=blue]{hyperref}


\title{半捻り関手の一般論}
\author{荒井 勇人}
\date{\today}
\begin{document}
\maketitle
\section{記法}
\begin{itemize}
    \item 導来関手の$R$と$L$は原則省略する。
    \item $X$が環付き空間のとき$\Mod(\cO_X)$を$\cO_X$加群の圏、$D(\cO_X)$を$\Mod(\cO_X)$の導来圏とする。$+, -, b$をつけて有界導来圏を表す。
    \item $X$がスキームのとき、$D_{\QCoh}(\cO_X) \subset D(\cO_X)$をコホモロジーが凖連接層であるような複体のなす充満部分圏とする。
    \item $X$がNoetherスキームのとき、$D_{\Coh}(\cO_X) \subset D(\cO_X)$をコホモロジーが連接層であるような複体のなす充満部分圏とする。また$D^b(X) = D_{\Coh}^b(\cO_X)$とする。($X$がNoetherなのでこれは連接層の圏$\Coh(X)$の導来圏と同値である。)
\end{itemize}
\section{Grothendieck--Verdier双対}
\begin{theorem}[{\cite[\href{https://stacks.math.columbia.edu/tag/0A9E}{Tag 0A9E}]{stacks-project}}]
    $f \colon X \to Y$をqcqsスキームの間の射とする。このとき
    \begin{equation}
        f_* \colon D_{\QCoh}(\cO_X) \to D_{\QCoh}(\cO_Y)
    \end{equation}
    は右随伴$a \colon D_{\QCoh}(\cO_Y) \to D_{\QCoh}(\cO_X)$を持つ。さらにこれは$D_{\QCoh}^+$を保つ。
\end{theorem}
\begin{definition}[{\cite[\href{https://stacks.math.columbia.edu/tag/0A74}{Tag 0A74}]{stacks-project}}]
    $i \colon Z \to X$をスキームの閉埋め込みとする。$\cO_X$加群$F$に対して、層$\CHom_{\cO_X}(i_*\cO_Z, F)$は自然に$\cO_Z$加群とみなせる。
    これにより定まる関手$\Mod(\cO_X) \to \Mod(\cO_Z)$を$\CHom(\cO_Z, -)$と書く。
\end{definition}
\begin{proposition}
    $i \colon Z \to X$をスキームの閉埋め込みとする。このとき
    \begin{enumerate}
        \item 関手$\CHom(\cO_Z, -) \colon \Mod(\cO_X) \to \Mod(\cO_Z)$は左完全である。
        \item その右導来関手を$R\CHom(\cO_Z, -) \colon D(\cO_X) \to D(\cO_Z)$とおくと、これは$(L i_* = )i_* = R i_* \colon D(\cO_Z) \to D(\cO_X)$の右随伴である。
        \item $R\CHom(\cO_Z, -)$は$D_{\QCoh}^+$を保つ。
        \item $i_* \colon D_{\QCoh}(\cO_Z) \to D_{\QCoh}(\cO_X)$の右随伴$a$と$R\CHom(\cO_Z, -)$は$D_{\QCoh}^+$へ制限したときに一致する。
    \end{enumerate}
\end{proposition}
\section{射影公式}
\begin{proposition}[射影公式]\label{prop:projection-formula-for-modules}
    $f \colon X \to Y$を環付き空間の射とし、$E \in D(\cO_X), K \in D(\cO_Y)$とする。このとき自然な射
    \begin{equation}
        f_*E \otimes K \to f_*(E \otimes f^*K)
    \end{equation}
    は以下の状況で同型である。
    \begin{enumerate}
        \item $K$がperfect({\cite[\href{https://stacks.math.columbia.edu/tag/0B54}{Tag 0B54}]{stacks-project}})
        \item $f$が閉集合への同相({\cite[\href{https://stacks.math.columbia.edu/tag/0B55}{Tag 0B55}]{stacks-project}})
    \end{enumerate}
\end{proposition}
\section{Fourier--Mukai変換}
$X, Y$を$S$上のスキームとし、$K \in D(\cO_{X \times_S Y})$とする。
\begin{equation}
    \Phi_K = p_{Y*}(p_{X,*}(-)\otimes_{\cO_{X \times_S Y}} K) \colon D(\cO_X) \to D(\cO_Y)
\end{equation}
を$K$に付随するFourier--向井関手という。$D_{\QCoh}$などの部分圏の間の関手になっているとき、それもFourier--向井関手という。

\begin{lemma}[{\cite[\href{https://stacks.math.columbia.edu/tag/0FYR}{Tag 0FYR}]{stacks-project}}]
    $X \to S$がqcqs射で$K \in D_{\QCoh}$なら$\Phi_K \colon D_{\QCoh}(\cO_X) \to D_{\QCoh}(\cO_Y)$である。
\end{lemma}
\begin{lemma}[{\cite[\href{https://stacks.math.columbia.edu/tag/0FYS}{Tag 0FYS}]{stacks-project}}]
    $X, Y, Z$が$S$上qcqsなスキームで、$X$と$Z$が$S$上tor independentかつ$Y$が$S$上平坦だとする。$K, K' \in D_{\QCoh}$のとき
    \begin{equation}
        K'' = p_{XZ,*}(p_{XY}^*K \otimes p_{YZ}^*K'')
    \end{equation}
    とおくと、
    \begin{equation}
        \Phi_{K'} \circ \Phi_K = \Phi_{K''}\colon D_{\QCoh}(\cO_X) \to D_{\QCoh}(\cO_Z)
    \end{equation}
    である。
\end{lemma}
\begin{example}\label{ex:composition-of-Fourier-Mukai-over-field}
    $S$が体$k$で$X, Y, Z$が$k$上qcqsなら使える。
\end{example}
\begin{example}\label{ex:push-pull-as-Fourier-Mukai}
    $f \colon X \to Y$を$S$上分離的なスキームの間の射とする。このとき以下で定まる$f$のグラフ$\Gamma_f \colon X \to X \times_S Y$は閉埋め込みである。
    \[
        \begin{tikzcd}
            X \ar[rd, "\Gamma_f"] \ar[rdd, bend right, "\id_X"] \ar[rrd, bend left, "f"]& & \\
            & X \times_S Y \ar[r] \ar[d] & Y \ar[d]\\
            & X            \ar[r]        & S
        \end{tikzcd}
    \]
    そして
    \begin{align}
        \Phi_{\cO_{\Gamma_f}}^{X \to Y} = f_* & \colon D(\cO_X) \to D(\cO_Y) \\
        \Phi_{\cO_{\Gamma_f}}^{Y \to X} = f_* & \colon D(\cO_Y) \to D(\cO_X)
    \end{align}
    である。
\end{example}
\begin{proof}
    $\Gamma_f$は閉埋め込みなので、\ref{prop:projection-formula-for-modules}より射影公式が使える。
    よって
    \begin{align}
        \Phi_{\cO_{\Gamma_f}}^{X \to Y}(-) & = p_{Y*}(p_X^*(-) \otimes \cO_{\Gamma_f})               \\
                                           & = p_{Y*}(p_X^*(-) \otimes \Gamma_{f*}\cO_X)             \\
                                           & = p_{Y*}\Gamma_{f*}(\Gamma_{f}^*p_X^*(-) \otimes \cO_X) \\
                                           & = f_*(-)
    \end{align}
    \begin{align}
        \Phi_{\cO_{\Gamma_f}}^{X \to Y}(-) & = p_{X*}(p_Y^*(-) \otimes \cO_{\Gamma_f})               \\
                                           & = p_{X*}(p_Y^*(-) \otimes \Gamma_{f*}\cO_X)             \\
                                           & = p_{X*}\Gamma_{f*}(\Gamma_{f}^*p_Y^*(-) \otimes \cO_X) \\
                                           & = f^*(-)
    \end{align}
    となる。
\end{proof}
\begin{example}\label{ex:line-bundle-as-Fourier-Mukai}
    $X$を$S$上分離的なスキームとし、$L$を$X$上の可逆層とする。このとき
    \begin{equation}
        \Phi_{\Delta_*L}(-) = L \otimes f^*(-) \colon D(\cO_X) \to D(\cO_Y)
    \end{equation}
    である。
\end{example}
\begin{proof}
    $\Delta$は閉埋め込みなので、\ref{prop:projection-formula-for-modules}より射影公式が使える。
    よって
    \begin{align}
        \Phi_{\Delta_*L}(-) & = p_{Y*}(p_X^*(-) \otimes \Delta_* L)        \\
                            & = p_{Y*}\Delta_*(\Delta^*p_X^*(-) \otimes L) \\
                            & = (-) \otimes L
    \end{align}
    である。
\end{proof}

\section{Cartier因子}
\begin{definition}
    $X$をスキームとする。閉部分スキーム$D \subset X$が有効Cartier因子であるとは、$D$の定義イデアル$I$が可逆層であることをいう。
    このとき$\cO_X(-D) = I, \cO_X(D) = \cO_X(-D)^{-1}$と定める。
    定義から完全列
    \begin{equation}
        0 \to \cO_X(-D) \to \cO_X \to \cO_D \to 0
    \end{equation}
    が存在する。

\end{definition}
\begin{proposition}
    $X$をスキームとし、$D \subset X$を閉部分スキームとする。このとき以下は同値である。
    \begin{enumerate}
        \item $D$は有効Cartier因子である。
        \item $D$の定義イデアルは局所的に非零因子で生成される。
    \end{enumerate}
\end{proposition}
\begin{proposition}
    $X$をNoetherスキームとし、$i \colon D\hookrightarrow X$を有効Cartier因子とする。
    このとき関手
    \begin{equation}
        i^*(-)\otimes \cO_D(D)[-1] \colon D_{\QCoh}^+(\cO_X) \to D_{\QCoh}^+(\cO_D)
    \end{equation}
    は関手$i_* \colon D_{\QCoh}^+(\cO_D) \to D_{\QCoh}^+(\cO_X)$の右随伴である。
    特に$i_* \colon D_{\QCoh}(\cO_D) \to D_{\QCoh}(\cO_X)$の右随伴$a$の$D_{\QCoh}^+$への制限と一致する。
\end{proposition}
\begin{proof}
    {\cite[\href{https://stacks.math.columbia.edu/tag/0a9X}{Tag 0a9X}]{stacks-project}}と{\cite[\href{https://stacks.math.columbia.edu/tag/0AA4}{Tag 0AA4}]{stacks-project}}より従う。
\end{proof}

\begin{proposition}
    $k$を体とし、$X$を$k$上分離的なNoetherスキーム、$i \colon D \hookrightarrow X$を有効Cartier因子とする。
    このとき任意の$F \in D_{\QCoh}^+(\cO_D)$に対して、$D(\cO_Y)$の完全三角形
    \begin{equation}
        F \otimes \cO_D(-D)[1] \to i^*i_*F \to F \xrightarrow{+1}
    \end{equation}
    が存在する。ここで$i^i_*F \to F$は随伴$i^* \dashv i_*$の余単位射、$F \otimes \cO_D(-D)[1] \to i^*i_*F$は
    \begin{equation}
        \Hom(F \otimes \cO_D(-D)[1], i^*i_*F) \cong \Hom(F, i^*i_*F\otimes \cO_D(-D)[1]) = \Hom(F, a(i_*F))
    \end{equation}
    によって随伴$i_* \dashv a$の単位射に対応するものである。

    $F = \cO_D$のときはより強く、
    \begin{equation}
        i^*i_*\cO_D \cong \cO_D \oplus \cO_D(-D)[-1]
    \end{equation}
    が成り立つ。
\end{proposition}
\begin{proof}
    \cite[corollary 11.4]{MR2244106}の証明がこの場合に使えることを確かめる。

    まず$F = \cO_D$のときに示す。
    完全列
    \begin{equation}\label{eq:koszul-resolution}
        0 \to \cO_X(-D) \to \cO_X \to i_*\cO_D \to 0
    \end{equation}
    が$i_*\cO_D$の平坦分解を与えるから、これにより$i^*$を計算すればよい。

    次に$F$が(一般の複体ではなく)$\cO_X$加群のときに成立することを示す。
    射影公式よりと分解\eqref{eq:koszul-resolution}より
    \begin{align}
        i_*i^*i_*F & \cong i_*\cO_D \otimes i_*F                                     \\
                   & \cong (0 \to i_*F\otimes\cO_X(-D) \xrightarrow{d} i_*F \to 0)   \\
                   & \cong (0 \to i_*(F\otimes\cO_D(-D)) \xrightarrow{d} i_*F \to 0)
    \end{align}
    となり、微分$d$は$D$に制限されているので$0$となる。よって$i^*i_*F$のコホモロジーは($i_*$してから計算できるので)
    \[
        \cH^j(i^*i_*F) \cong \begin{cases}
            F                   & (j = 0)            \\
            F \otimes \cO_D(-D) & (j = -1)           \\
            0                   & (\text{otherwise})
        \end{cases}
    \]
    となる。
    特に$i^*i_*F$は2項複体だから、完全三角形
    \begin{equation}
        \cH^{-1}[1] \to i^*i_*F \to \cH^0 \xrightarrow{+1}
    \end{equation}
    が存在する。
    あとは
    右端のコホモロジーへの自然な射$i^*i_*F \to \cH^0 \cong F$が随伴$i^*\dashv i_*$の余単位射に、
    左端のコホモロジーからの自然な射$F \otimes \cO_D(-D)[1] \cong \cH^{-1}[1] \to i^*i_*F$が随伴$i_* \dashv a$の単位射に
    対応することを示せばよい。
    $P^\bullet \to i_*F$と$i_*F \to I^\bullet$をそれぞれ$i_*F$の平坦分解と単射分解とする。
    このとき$\id_{i_*F}$は同型
    \begin{equation}
        \Hom(i_*F, i_*F) \cong \Hom(P^\bullet, i_*F) \cong \Hom(i^*P^\bullet, F)
    \end{equation}
    を通して$P^\bullet \to i_*F$の随伴
    \[
        \begin{tikzcd}
            \cdots \ar[r] & i^*P^{-1} \ar[r] \ar[d] & i^*P^0 \ar[r] \ar[d] & 0 \ar[d]\ar[r]& \cdots \\
            \cdots \ar[r] & 0 \ar[r]     & F \ar[r]     & 0\ar[r]& \cdots
        \end{tikzcd}
    \]
    に対応する。これは右端のコホモロジーへの自然な射に他ならない。
    同様に同型
    \begin{equation}
        \Hom(i_*F, i_*F) \cong \Hom(i_*F, I^\bullet) \cong \Hom(F, a(I^\bullet)) \cong \Hom(F, R\CHom(\cO_D, I^\bullet))
    \end{equation}
    により$\id_{i_*F}$は
    \[
        \begin{tikzcd}
            \cdots \ar[r] & 0 \ar[r]  \ar[d]    & F \ar[r]\ar[d]     & 0\ar[r]\ar[d]& \cdots\\
            \cdots \ar[r] & 0 \ar[r]  & R\CHom(\cO_D, I^0) \ar[r] & R\CHom(\cO_D, I^1)\ar[r]& \cdots
        \end{tikzcd}
    \]
    に対応し、これは左端のコホモロジーからの自然な射に他ならない。

    最後一般の$F \in D_{\QCoh}^+(\cO_D)$に対して示す。
    $i \colon D \hookrightarrow X$のグラフを$\Gamma \subset D \times X$、その転置を$\overline{\Gamma} \subset X \times D$とする。
    \[
        \begin{tikzcd}
            && D &\\
            &\Gamma \times D \ar[r, hook, "\varphi"]\ar[d, "\pi"] &D \times D \times D \ar[d, hook, "\psi"] \ar[u, "\pi_2"] & D \times \overline{\Gamma} \ar[l, hook', "\overline{\varphi}"]\\
            D  \ar[r, hook, "\Delta"] \ar[ru, bend left, "\eta"]&D \times D & D \times X \times D \ar[l, "\pi_{DD}"] &
        \end{tikzcd}
    \]
    という可換図式を考える。ただし$\eta$は$D \xrightarrow{\sim} \Gamma$と$\id_D$の積である。
    有効Cartier因子$\varphi \colon D \times D \times D \hookrightarrow D \times X \times D$と$\cO_{D \times D \times D}$加群$\overline{\varphi}_*\cO_{D \times \overline{\Gamma}}$に対して命題を適用すると、完全三角形
    \begin{equation}\label{eq:triangle-on-DDD}
        \overline{\varphi}_* \cO_{D \times \overline{\Gamma}}\otimes \pi_2^*\cO_D(-D)[1]\to \psi^*\psi_*\overline{\varphi}_*\cO_{D \times \overline{\Gamma}}\to \overline{\varphi}_*\cO_{D \times \overline{\Gamma}} \xrightarrow{+1}
    \end{equation}
    を得る。
    $\eta \colon D \hookrightarrow D \times D \times D$を対角射とすると、
    \begin{equation}
        \varphi^*\overline{\varphi}_*\cO_{X \times \overline{\Gamma}} \cong \eta_*\cO_D
    \end{equation}
    かつ
    \begin{equation}
        \varphi^*(\overline{\varphi}_* \cO_{D \times \overline{\Gamma}}\otimes \pi_2^*\cO_D(-D)[1]) \cong \eta_*\cO_D(-D)[1]
    \end{equation}
    だから、\eqref{eq:triangle-on-DDD}を$\varphi$で引き戻すことで$D_{\QCoh}(\cO_{\Gamma \times D})$の完全三角形
    \begin{equation}
        \eta_*\cO_D(-D)[1] \to \varphi^*\psi^*\psi_*\overline{\varphi}_*\cO_{D \times \overline{\Gamma}} \to \eta_*\cO_D \xrightarrow{+1}
    \end{equation}
    を得る。
    これをさらに$\pi$で押し出すと
    \begin{equation}
        \Delta_*\cO_D(-D)[1] \to \pi_*\varphi^*\psi^*\psi_*\overline{\varphi}_*\cO_{D \times \overline{\Gamma}} \to \Delta_*\cO_D \xrightarrow{+1}
    \end{equation}
    という$D_{\QCoh}(\cO_{D \times D})$の完全三角形となる。
    ここで
    \begin{align}
        \pi_*\varphi^*\psi^*\psi_*\overline{\varphi}_*\cO_{D \times \overline{\Gamma}} & \cong \pi_{DD}((\psi \circ \varphi)_*\cO_{\Gamma \times D} \otimes (\psi \circ \overline{\varphi})_*\cO_{D \times \overline{\Gamma}}) \\
                                                                                       & \cong \pi_{DD}(\pi_{DX}^*\cO_{\Gamma} \otimes \pi_{XD}\cO_{\overline{\Gamma}})
    \end{align}
    が成り立つ。1つ目の同型は$\psi \circ \varphi$についての射影公式(命題\ref{prop:projection-formula-for-modules}(2))である。
    以上より$D_{\QCoh}(\cO_{D \times D})$の完全三角形
    \begin{equation}\label{eq:triangle-on-DD}
        \Delta_*\cO_D(-D)[1] \to \pi_{DD}(\pi_{DX}^*\cO_{\Gamma} \otimes \pi_{XD}\cO_{\overline{\Gamma}}) \to \Delta_*\cO_D \xrightarrow{+1}
    \end{equation}
    を得る。これらは例\ref{ex:line-bundle-as-Fourier-Mukai},\ref{ex:composition-of-Fourier-Mukai-over-field},\ref{ex:push-pull-as-Fourier-Mukai}より(ここで体上であることと分離性を使う)それぞれ$(-)\otimes \cO_D(-D)[1], i^*i_*, \id$のFourier--向井核だから、命題の完全三角形を得る。
    完全三角形の中の射の説明についても三角形\eqref{eq:triangle-on-DDD}の構成からわかる。
\end{proof}

\section{共役作用}
\begin{proposition}
    $i \colon Y \hookrightarrow X$を$X \to T$のファイバーとし、$E \in D^b(Y)$を$i$についての半球面対象とする。
    さらに$L \in \Pic(Y)$を、$E \otimes L$も半球面対象となるような直線束とする。
    このとき
    \begin{equation}
        (L \otimes -) \circ H_E \circ (L \otimes -)^{-1} \cong H_{E\otimes L}
    \end{equation}
    が成り立つ。
\end{proposition}
\begin{proof}
    % TODO
\end{proof}
\printbibliography[title=参考文献]
\end{document}