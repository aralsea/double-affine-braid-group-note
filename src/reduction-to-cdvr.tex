\RequirePackage{plautopatch}
\documentclass[uplatex, a4paper, dvipdfmx]{jsarticle}
\usepackage[backend=biber, style=alphabetic, sorting=nyt]{biblatex}
\addbibresource{../bib/double-affine-braid-group.bib} 

\usepackage{amsthm}
\usepackage{amsmath,amsfonts,amssymb}
\usepackage{tikz-cd}
\usepackage{dynkin-diagrams}

\renewcommand{\labelenumi}{(\arabic{enumi})}

\theoremstyle{definition}
\newtheorem{theorem}{定理}[section]
\newtheorem{definition}[theorem]{定義}
\newtheorem{proposition}[theorem]{命題}
\newtheorem{lemma}[theorem]{補題}
\newtheorem{corollary}[theorem]{系}
\newtheorem{fact}[theorem]{事実}
\newtheorem{example}[theorem]{例}
\newtheorem{remark}[theorem]{補足}

\DeclareMathOperator{\Hom}{\mathrm{Hom}}
\DeclareMathOperator{\Tor}{\mathrm{Tor}}
\DeclareMathOperator{\CHom}{\mathcal{H}\!\mathit{om}}
\DeclareMathOperator{\CTor}{\mathcal{T}\!\mathit{or}}
\DeclareMathOperator{\Auteq}{\mathrm{Auteq}}
\DeclareMathOperator{\Cone}{\mathrm{Cone}}
\DeclareMathOperator{\ev}{\mathrm{ev}}
\DeclareMathOperator{\id}{\mathrm{id}}
\DeclareMathOperator{\depth}{\mathrm{depth}}
\DeclareMathOperator{\Pic}{\mathrm{Pic}}
\DeclareMathOperator{\MCG}{\mathrm{MCG}}
\DeclareMathOperator{\PMCG}{\mathrm{PMCG}}
\DeclareMathOperator{\RHom}{\mathrm{RHom}}
\DeclareMathOperator{\Ker}{\mathrm{Ker}}
\DeclareMathOperator{\Coker}{\mathrm{Coker}}
\DeclareMathOperator{\Image}{\mathrm{Im}}
\DeclareMathOperator{\Aut}{\mathrm{Aut}}
\DeclareMathOperator{\Inn}{\mathrm{Inn}}
\DeclareMathOperator{\Out}{\mathrm{Out}}
\DeclareMathOperator{\Supp}{\mathrm{Supp}}
\DeclareMathOperator{\SL}{\mathrm{SL}}
\DeclareMathOperator{\GL}{\mathrm{GL}}
\DeclareMathOperator{\Spec}{\mathrm{Spec}}
\DeclareMathOperator{\Proj}{\mathrm{Proj}}
\DeclareMathOperator{\Perf}{\mathrm{Perf}}
\DeclareMathOperator{\NS}{\mathrm{NS}}
\DeclareMathOperator{\Ext}{\mathrm{Ext}}
\DeclareMathOperator{\Hilb}{\mathrm{Hilb}}
\DeclareMathOperator{\res}{\mathrm{res}}
\DeclareMathOperator{\Ch}{\mathrm{Ch}}
\DeclareMathOperator{\coh}{\mathrm{coh}}
\DeclareMathOperator{\Ann}{\mathrm{Ann}}
\DeclareMathOperator{\Aff}{\mathrm{Aff}}
\DeclareMathOperator{\FM}{\mathrm{FM}}
\DeclareMathOperator{\red}{\mathrm{red}}
\DeclareMathOperator{\length}{\mathrm{length}}
\DeclareMathOperator{\Div}{\mathrm{Div}}
\DeclareMathOperator{\DivCl}{\mathrm{DivCl}}
\DeclareMathOperator{\Cl}{\mathrm{Cl}}
\DeclareMathOperator{\Sch}{\mathrm{Sch}}
\DeclareMathOperator{\Set}{\mathrm{Set}}
\DeclareMathOperator{\op}{\mathrm{op}}
\DeclareMathOperator{\Coh}{\mathrm{Coh}}
\DeclareMathOperator{\QCoh}{\mathrm{QCoh}}
\DeclareMathOperator{\Mod}{\mathrm{Mod}}




\newcommand{\nc}{\newcommand}

%% Calligraphic letters

\nc{\cF}{{\mathcal{F}}}
\nc{\cG}{{\mathcal{G}}}
\nc{\cH}{{\mathcal{H}}}
\nc{\cJ}{{\mathcal{J}}}
\nc{\cM}{{\mathcal{M}}}
\nc{\cO}{{\mathcal{O}}}
\nc{\cU}{{\mathcal{U}}}
\nc{\cW}{{\mathcal{W}}}

%% Blackboard letters
\nc{\bA}{{\mathbb{A}}}
\nc{\bC}{{\mathbb{C}}}
\nc{\bG}{{\mathbb{G}}}
\nc{\bP}{{\mathbb{P}}}
\nc{\bQ}{{\mathbb{Q}}}
\nc{\bR}{{\mathbb{R}}}
\nc{\bZ}{{\mathbb{Z}}}


%% Fraktur letters
\nc{\fg}{{\mathfrak{g}}}
\nc{\fp}{{\mathfrak{p}}}
\nc{\fu}{{\mathfrak{u}}}

% hyperref
\usepackage[urlcolor=blue]{hyperref}


\title{メモ}
\author{荒井 勇人}
\date{\today}
\begin{document}
\maketitle
\section{記法}
\begin{itemize}
    \item $S, X, Y, Z$はすべてスキームとする。
    \item $X$に対し、$\cO_X$加群のAbel圏の導来圏を$D(\cO_X)$、そのうち準連接なコホモロジーを持つ複体の充満部分圏を$D_{\QCoh}(\cO_X)$と書く。
    \item $S$上のスキーム$X, Y$と$P \in D(\cO_{X \times_S Y})$に対し、$$\Phi_P(-) = \Phi_P^{X \to Y}(-) = \pi_{Y*}(P \otimes \pi_X^*(-)) \colon D(\cO_X) \to D(\cO_Y)$$を$P$を核とする積分関手とする。
\end{itemize}

\section{積分関手とその合成}
\begin{definition}
    $X, Y, Z$を$S$上平坦かつqcqsなスキームとする。このとき{\cite[\href{https://stacks.math.columbia.edu/tag/0FYS}{Tag 0FYS}]{stacks-project}}より、$P \in D_{\QCoh}(\cO_{X \times_S Y}), P' \in D_{\QCoh}(\cO_{Y \times_S Z})$のとき
    \begin{equation}
        P'' = \pi_{XZ*}(\pi_{XY}^*P \otimes \pi_{YZ}^*P')
    \end{equation}
    とおくと、
    \begin{equation}
        \Phi_{P'} \circ \Phi_P = \Phi_{P''}\colon D_{\QCoh}(\cO_X) \to D_{\QCoh}(\cO_Z)
    \end{equation}
    である。この$P''$を$P' * P$と書く。
\end{definition}
\begin{proposition}[押し出しと引き戻し]\label{ex:push-pull-as-integral-functor}
    $X, Y$を$S$上分離的なスキーム、$f \colon X \to Y$を$S$上の射とする。$f$のグラフを$\Gamma_f \subset X \times_S Y$とすると、
    \begin{align}
        \Phi_{\cO_{\Gamma_f}}^{X \to Y} = f_* & \colon D(\cO_X) \to D(\cO_Y) \\
        \Phi_{\cO_{\Gamma_f}}^{Y \to X} = f_* & \colon D(\cO_Y) \to D(\cO_X)
    \end{align}
    である。
\end{proposition}
\begin{proof}
    射影公式がスキームの任意の閉埋め込みについて成り立つこと({\cite[\href{https://stacks.math.columbia.edu/tag/08EU}{Tag 08EU}]{stacks-project}})と、分離性より$\Gamma_f \subset X \times_S Y$が閉埋め込みであることを用いると、普通の証明が回ることがわかる。
\end{proof}
\begin{proposition}
    $X, Y, Z$を$S$平坦かつqcかつ分離的なスキームとする。
    \begin{enumerate}
        \item $f \colon X \to Y$が$S$上の射で$P \in D_{\QCoh}(\cO_{Y \times_S Z})$のとき、$P * \cO_{\Gamma_f} \cong (f \times \id_Z)^*P \in D_{\QCoh}(\cO_{X \times_S Z})$である。
        \item $P \in D_{\QCoh}(\cO_{X \times_S Y})$で$g \colon Y \to X$が$S$上の射のとき、$\cO_{\Gamma_g} * P \cong (\id_Z \times g)_*P \in D_{\QCoh}(\cO_{X \times_S Z})$である。
    \end{enumerate}
\end{proposition}
\begin{proof}
    以下$\Gamma_f, \Gamma_g$を$\Gamma$と略記する。$\times_S$も$\times$と略記する。
    \begin{enumerate}
        \item \[
                  \begin{tikzcd}
                      & Y \times Z\\
                      X \times Z \ar[r, "\varphi"] \ar[rd, bend right, "\id"] \ar[ru, "f \times \id", bend left]& X \times Y \times Z \ar[d, "\pi_{XZ}"] \ar[u, "\pi_{YZ}"]\\
                      &X \times Z
                  \end{tikzcd}
              \]
              $\varphi \colon X \times Z \to X \times Y \times Z$を$\varphi(x, z) = (x, f(x), z)$と定めると、上のような可換図式を得る。
              \begin{align}
                  P * \cO_\Gamma & = \pi_{XZ*}(\pi_{XY}^*\cO_{\Gamma} \otimes \pi_{YZ}^*P)        \\
                                 & \cong \pi_{XZ*}(\cO_{\Gamma \times Z}\otimes\pi_{YZ}^*P)       \\
                                 & \cong \pi_{XZ*}(\varphi_*(\cO_{X \times Z})\otimes\pi_{YZ}^*P) \\
                                 & \cong \pi_{XZ*}\varphi_*\varphi^*\pi_{YZ}^*P                   \\
                                 & \cong (f \times \id_Z)^*P
              \end{align}
              となる。(7)は$\varphi$と$\Gamma$の定義、(8)は射影公式である。
        \item 同様。
    \end{enumerate}
\end{proof}

\section{完備局所環への帰着}
\begin{itemize}
    \item $k$を標数$0$の代数閉体とする。
    \item 登場するスキームはすべて$k$上のものとする。
    \item $D^b(X)$を$X$上の連接層の有界導来圏とする。
    \item 導来関手の$L, R$は基本的に省略する。
\end{itemize}

\begin{definition}
    $\pi \colon X \to T$を非特異準射影多様体の間の平坦射とする。
    $i \colon Y \hookrightarrow X$をの閉点$0 \in T$でのファイバーとし、$k \colon Y \times Y \hookrightarrow X \times_T X$とする。$E \in D^b(Y)$が$i_*$で送って$D^b(X)$の球面対象になっているとき、以下のようにして$D^b(Y)$の自己同値$H_E$が定まる。
    \begin{enumerate}
        \item $E' = \CHom(E, i^!\cO_X) \cong \CHom(E, \cO_Y[-1])$とする。
        \item 自然な射$k_*(E' \boxtimes E) \to \cO_{\Delta} \in D(\cO_{X \times_T X})$がある。
        \item $P = \Cone(k_*(E' \boxtimes E) \to \cO_{\Delta})$とする。
        \item $H_E = \Phi_{k^*P} \colon D^b(Y) \to D^b(Y)$とする。
    \end{enumerate}
\end{definition}
この節の目的は、$X \to T$を$\Spec \widehat{\cO_{T, 0}}$でbase changeしても$H_E$が変わらないことを示すことである。以下$R = \cO_{T, 0}$とし、$\widehat{R}$を$R$の(極大イデアルによる)完備化とする。以下のように射を定める。
\[
    \begin{tikzcd}
        Y \ar[r, "f"] \ar[d]&X_{\widehat{R}} \ar[r, "g"] \ar[d]& X_R \ar[r, "h"] \ar[d]& X \ar[d]\\
        \{0\} \ar[r]& \Spec \widehat{R} \ar[r] & \Spec R \ar[r] & T
    \end{tikzcd}
\]
\[
    \begin{tikzcd}
        Y \times Y \ar[r, "f \times f"] \ar[rrr, "k", bend left]& X_{\widehat{R} \times_{\widehat{R}}} X_{\widehat{R}} \ar[r, "g \times g"] & X_R \times_R X_R \ar[r, "h \times h"] & X \times_T X
    \end{tikzcd}
\]
以下のことを示したい。
\begin{proposition}
    $P_{\widehat{R}} = \Cone((f \times f)_*(E' \boxtimes E) \to \cO_{\Delta_{X_{\widehat{R}}}}) \in D_{\QCoh}(\cO_{X_{\widehat{R} \times_{\widehat{R}}} X_{\widehat{R}}})$とすると、$P_{\widehat{R}} \cong (g \times g)^*(h \times h)^*P$である。
\end{proposition}
そのために以下を示す必要がある。
\begin{itemize}
    \item $(f \times f)_*(E' \boxtimes E) \cong (g \times g)^*(h \times h)^*k_*(E' \boxtimes E)$である。
    \item $\cO_{\Delta_{X_{\widehat{R}}}} \cong (g \times g)^*(h \times h)^*\cO_{\Delta_X}$である。
    \item 上の2つの同型により、射
          \begin{align}(f \times f)_*(E' \boxtimes E)                  & \to \cO_{\Delta_{X_{\widehat{R}}}},            \\
             (g \times g)^*(h \times h)^*k_*(E' \boxtimes E) & \to (g \times g)^*(h \times h)^*\cO_{\Delta_X}\end{align}
          が対応する。
\end{itemize}

% \begin{remark}
%     この命題をもって$X$を完全に$X_{\widehat{R}}$に置き換えてしまうのは難しいと思われる。なぜなら、(私が)球面捻りを$X_{\widehat{R}}$の導来圏で定式化できていないからである(実はできるのかもしれない)。例えば\cite{MR3692883}による定式化では積分関手の随伴が必要であり、そのためには$p \colon X_{\widehat{R}} \to \Spec k$についてGrothendieck双対が必要である。しかし現状知られているGrothendieck双対の最も一般的な設定は、(本質的)有限型な射についてのものである。

%     アイデア:\cite[Theorem 3.1.1]{MR1996800}によると、qcqsスキーム$X$について$D_{\QCoh}(\cO_X)$はcompact generatorを持つ。\cite[Corollary 4.23]{MR2434186}によると、余完備でcompact generatorを持つ体上の三角圏はRouquier functorを持つ。これと\cite{MR1831820}の定式化を使って、$X_{\widehat{R}}$の導来圏の球面捻りを定式化できるかもしれない。
% \end{remark}

\subsection{1つ目の証明}
$k = (h \times h) \circ (g \times g) \circ (f \times f)$だから、1つ目を示すには$(h \times h)^*(h \times h)_* \cong \id, (g \times g)^*(g \times g)_* \cong \id$を示せばよい。
\subsubsection{局所化について}
\begin{lemma}\label{lem:base-change-by-localizaion}
    $X, T$をスキーム、$t \in T$とし、以下のbase changeの図式を考える。
    \[
        \begin{tikzcd}
            X \times_T \Spec \cO_{T, t} \ar[r, "i"] \ar[d]& X \ar[d]\\
            \Spec \cO_{T, t} \ar[r]& T
        \end{tikzcd}
    \]
    このとき
    \begin{enumerate}
        \item $i$は平坦である。
        \item $i_* \colon \QCoh(X \times_T \Spec \cO_{T, t}) \to \QCoh(X)$は完全関手である。
        \item (随伴の余単位射によって)$i^*i_* \cong \id \colon \QCoh( X \times_T \Spec \cO_{T, t}) \to \QCoh( X \times_T \Spec \cO_{T, t})$である。
    \end{enumerate}
\end{lemma}
\begin{proof}
    \begin{enumerate}
        \item $\Spec \cO_{T, t} \to T$が平坦だからよい。
        \item 開埋め込みによる押し出しが完全なので、$T$をアファインに置き換えてよい。さらに$X$について局所的に確かめればよいので、$X = \Spec A, T = \Spec R, t = \fp, X \times_T \Spec \cO_{T, t} = \Spec A_{\fp}$としてよい。このとき$i_*$は係数忘却する関手$\Mod(A_p) \to \Mod(A)$と同一視でき、これは完全である。
        \item (2)と同様に$X = \Spec A, T = \Spec R, t = \fp, X \times_T \Spec \cO_{T, t} = \Spec A_{\fp}$としてよい。すると$i^*i_*$は$(-)\otimes_A A_{\fp} \colon \Mod(A_{\fp}) \to \Mod(A_{\fp})$と同一視でき、これは$\id$である。
    \end{enumerate}
\end{proof}
\begin{lemma}
    補題\ref{lem:base-change-by-localizaion}の設定で、$X, T$はNoetherとし$Y =  X \times_T \Spec \cO_{T, t}$とおく。
    このとき(随伴の余単位によって)
    \begin{equation}
        Li^* \circ Ri_* \cong \id \colon D_{\QCoh}(\cO_Y) \to D_{\QCoh}(\cO_Y)
    \end{equation}
    である。($D(\cO_X)$でも正しいかは不明。)
\end{lemma}
\begin{proof}
    ここでは導来関手の$L, R$は省略しない。

    まず$i_* \colon \QCoh(Y) \to \QCoh(Y)$の右導来関手は、単に複体の各項に$i_*$をあてる関手$i_* \colon D(\QCoh(X)) \to D(\QCoh(Y))$である。
    \cite[\href{https://stacks.math.columbia.edu/tag/09T3}{Tag 09T3}]{stacks-project}(ここにNoether性を使う)と合わせると、横向きの関手が自然な圏同値であるような可換図式
    \[
        \begin{tikzcd}
            D(\QCoh(Y)) \ar[r] \ar[d, "i_*"] & D_{\QCoh}(\cO_Y) \ar[d, "Ri_*"]\\
            D(\QCoh(X)) \ar[r]& D_{\QCoh}(\cO_X) \ar[d, "Li^*"]\\
            & D_{\QCoh}(\cO_Y)
        \end{tikzcd}
    \]
    を得る。$i$は平坦だから、$Li^*$も複体の各項に$i^*$をあてる関手である。よって図式の横向きの関手$D(\QCoh(Y)) \to D_{\QCoh}(\cO_Y)$に$Li^* \circ Ri_*$を合成したものは、再び$D(\QCoh(Y)) \to D_{\QCoh}(\cO_Y)$になる。よって$Li^* \circ Ri_* \cong \id$である。
\end{proof}
この補題を$X \times_T X \to T$に適用することで$(h \times h)^*(h \times h)_* \cong \id$がわかる。

\subsubsection{完備化について}
局所化についての証明と同様のことをする。
\begin{lemma}\label{lem:forget-and-extension}
    $R$をNoether局所環、$\widehat{R}$をその完備化、$M$を$\widehat{R}$加群とする。このとき自然な射
    \begin{align}
        M \to M \otimes_R \widehat{R} & ,\quad m \mapsto m \otimes 1                    \\
        M \otimes_R \widehat{R} \to M & ,\quad m \otimes \hat{r} \mapsto \hat{r}\cdot m
    \end{align}
    は互いに逆な同型である。
\end{lemma}
\begin{proof}
    まず$M = \widehat{R}$とする。このとき$\widehat{R}$は有限生成だから$\widehat{R} \otimes_R \widehat{R}\cong \widehat{\widehat{R}}$であり(ここでNoether性を使った)、完備性より$\widehat{\widehat{R}} \cong \widehat{R}$である。

    一般の$M$については完全列
    \begin{equation}
        0 \to N \to \widehat{R}^{\oplus I} \to M \to 0
    \end{equation}
    をとる。すると命題の2つ目の射は完全列の間の射
    \[
        \begin{tikzcd}
            &N \otimes_R \widehat{R} \ar[r]\ar[d]& \widehat{R}^{\oplus I} \otimes_R \widehat{R} \ar[r]\ar[d]& M \otimes_R \widehat{R} \ar[r]\ar[d]& 0\\
            0 \ar[r] &N \ar[r] & \widehat{R}^{\oplus I}  \ar[r] & M  \ar[r] & 0
        \end{tikzcd}
    \]
    を与える。定義より左の射は全射で、テンソル積が任意個数の直和と可換であることと$M = \widehat{R}$の場合より中央の射は同型となる。よって蛇の補題より右側の射は同型である。
\end{proof}
\begin{lemma}
    $R$をNoether局所環、$X$をNoetherスキーム、$X \to \Spec R$を任意の射とする。$R$の完備化を$\widehat{R}$とし、以下のbase changeの図式を考える。
    \[
        \begin{tikzcd}
            X_{\widehat{R}} \ar[r, "i"] \ar[d]& X \ar[d]\\
            \Spec \widehat{R} \ar[r]& \Spec R
        \end{tikzcd}
    \]
    このとき
    \begin{enumerate}
        \item $i$は平坦である。
        \item $i_* \colon \QCoh(X_{\widehat{R}}) \to \QCoh(X)$は完全関手である。
        \item $i^*i_* \cong \id \colon \QCoh(X_{\widetilde{R}}) \to  \QCoh(X_{\widetilde{R}})$である。
    \end{enumerate}
\end{lemma}
\begin{proof}
    (1)と(2)は局所化のときと同様である。(3)については、以下を示せばよい。
    \begin{itemize}
        \item $A$を$R$代数とし$M$を$A \otimes_R \widehat{R}$加群とする。このとき(自然な射により)$M \otimes_A (A \otimes_R \widehat{R}) \cong M$である。
    \end{itemize}
    $M \otimes_A (A \otimes_R \widehat{R}) \cong M \otimes_R \widehat{R}$だから、$M$を$\widehat{R}$加群とみなして補題\ref{lem:forget-and-extension}を使えばよい。
\end{proof}
あとは局所化と同様にして完備化についての主張がわかる。
\subsection{2つ目の証明}
flat base changeである。
\subsection{3つ目の証明}
\printbibliography[title=参考文献]
\end{document}