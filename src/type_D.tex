\RequirePackage{plautopatch}
\documentclass[uplatex, a4paper, dvipdfmx]{jsarticle}
\usepackage[backend=biber, style=alphabetic, sorting=nyt]{biblatex}
\addbibresource{../bib/double-affine-braid-group.bib} 

\usepackage{amsthm}
\usepackage{amsmath,amsfonts,amssymb}
\usepackage{tikz-cd}
\usepackage{dynkin-diagrams}

\renewcommand{\labelenumi}{(\arabic{enumi})}

\theoremstyle{definition}
\newtheorem{theorem}{定理}[section]
\newtheorem{definition}[theorem]{定義}
\newtheorem{proposition}[theorem]{命題}
\newtheorem{lemma}[theorem]{補題}
\newtheorem{corollary}[theorem]{系}
\newtheorem{fact}[theorem]{事実}
\newtheorem{example}[theorem]{例}
\newtheorem{remark}[theorem]{補足}

\DeclareMathOperator{\Hom}{\mathrm{Hom}}
\DeclareMathOperator{\Tor}{\mathrm{Tor}}
\DeclareMathOperator{\CHom}{\mathcal{H}\!\mathit{om}}
\DeclareMathOperator{\CTor}{\mathcal{T}\!\mathit{or}}
\DeclareMathOperator{\Auteq}{\mathrm{Auteq}}
\DeclareMathOperator{\Cone}{\mathrm{Cone}}
\DeclareMathOperator{\ev}{\mathrm{ev}}
\DeclareMathOperator{\id}{\mathrm{id}}
\DeclareMathOperator{\depth}{\mathrm{depth}}
\DeclareMathOperator{\Pic}{\mathrm{Pic}}
\DeclareMathOperator{\MCG}{\mathrm{MCG}}
\DeclareMathOperator{\PMCG}{\mathrm{PMCG}}
\DeclareMathOperator{\RHom}{\mathrm{RHom}}
\DeclareMathOperator{\Ker}{\mathrm{Ker}}
\DeclareMathOperator{\Image}{\mathrm{Im}}
\DeclareMathOperator{\Aut}{\mathrm{Aut}}
\DeclareMathOperator{\Inn}{\mathrm{Inn}}
\DeclareMathOperator{\Out}{\mathrm{Out}}
\DeclareMathOperator{\Supp}{\mathrm{Supp}}
\DeclareMathOperator{\SL}{\mathrm{SL}}
\DeclareMathOperator{\GL}{\mathrm{GL}}
\DeclareMathOperator{\Spec}{\mathrm{Spec}}
\DeclareMathOperator{\Proj}{\mathrm{Proj}}
\DeclareMathOperator{\Perf}{\mathrm{Perf}}
\DeclareMathOperator{\NS}{\mathrm{NS}}
\DeclareMathOperator{\Ext}{\mathrm{Ext}}
\DeclareMathOperator{\Hilb}{\mathrm{Hilb}}
\DeclareMathOperator{\res}{\mathrm{res}}
\DeclareMathOperator{\Ch}{\mathrm{Ch}}
\DeclareMathOperator{\coh}{\mathrm{coh}}
\DeclareMathOperator{\Ann}{\mathrm{Ann}}
\DeclareMathOperator{\Aff}{\mathrm{Aff}}
\DeclareMathOperator{\FM}{\mathrm{FM}}
\DeclareMathOperator{\red}{\mathrm{red}}
\DeclareMathOperator{\length}{\mathrm{length}}



\newcommand{\nc}{\newcommand}

%% Calligraphic letters

\nc{\cF}{{\mathcal{F}}}
\nc{\cG}{{\mathcal{G}}}
\nc{\cH}{{\mathcal{H}}}
\nc{\cJ}{{\mathcal{J}}}
\nc{\cM}{{\mathcal{M}}}
\nc{\cO}{{\mathcal{O}}}
\nc{\cU}{{\mathcal{U}}}
\nc{\cW}{{\mathcal{W}}}

%% Blackboard letters
\nc{\bA}{{\mathbb{A}}}
\nc{\bC}{{\mathbb{C}}}
\nc{\bP}{{\mathbb{P}}}
\nc{\bQ}{{\mathbb{Q}}}
\nc{\bR}{{\mathbb{R}}}
\nc{\bZ}{{\mathbb{Z}}}


%% Fraktur letters
\nc{\fg}{{\mathfrak{g}}}
\nc{\fu}{{\mathfrak{u}}}

% hyperref
\usepackage[urlcolor=blue]{hyperref}


\title{二重アファインブレイド群の作用}
\author{荒井 勇人}
\date{\today}
\begin{document}
\maketitle


\section{記法}
$(V,(-, -))$を$n$次元ユークリッド空間とし、$R$をその上の既約有限ルート系とする。
\begin{itemize}
    \item $\{\alpha_1, \dots, \alpha_n\} \subset R$を$R$の基底とする。つまり$\alpha_i$たちは単純ルートである。
    \item $\{\alpha_1^\vee, \dots, \alpha_n^\vee\} \subset R$を単純コルートの集合とする。つまり$\alpha^\vee = \frac{2\alpha}{(\alpha, \alpha)}$である。
    \item $Q = \bigoplus_{i = 1}^n \bZ \alpha_i \subset V$ をルート格子とし、 $Q^\vee = \bigoplus_{i = 1}^n \bZ \alpha_i^\vee \subset V$ をコルート格子とする。
    \item $P = \{\lambda \in V \mid (\lambda, Q^\vee) \subset \bZ\}$をウェイト格子とし、$P^\vee = \{\lambda' \in V \mid (\lambda', Q) \subset \bZ\}$をコウェイト格子とする。
    \item $s_\alpha$を$\alpha$に付随する鏡映、つまり$s_\alpha(x) = x - (x, \alpha)\alpha^\vee$で定まる$V$の(内積を保つ)自己同型とする。$R$のWeyl群$W = \langle s_\alpha \mid \alpha \in R \rangle$はそれらの生成する$\GL(V)$の部分群である。これは単純ルートに付随する鏡映$s_i = s_{\alpha_i}$たちで生成される有限群である。
\end{itemize}
\section{ルート系、Weyl群、ブレイド群}
$(V,(-, -))$を$n$次元ユークリッド空間とする。
\begin{itemize}
    \item $F=\Aff(V)$を$V$上の$\bR$値アファイン関数全体のなすベクトル空間とし、$V$の内積を使って$F \cong V \oplus \bR \delta$と同一視する。ここで$\delta$は定数$1$を返す関数である。$f \in F$のとき$V$成分、つまり勾配ベクトルを$Df \in V$で表す。
    \item $V$の内積を$\bR \delta$上$0$として$F$上の双線形形式に拡張する。そして定数でない$f \in F$について$f^\vee = \frac{2f}{(f, f)}$と定める。
    \item 定数でない$f \in F$について$f^{-1}(0) \subset V$をアファイン超平面とし、それについての鏡映を$s_f$とおく。その$F$への作用も$s_f$とかく。
    \item $v \in V$についての平行移動を$t(v)$とおく。
\end{itemize}

\begin{definition}
    $F$の部分集合$R^a$が$V$上のアファインルート系であるとは以下を満たすこと。
    \begin{enumerate}
        \item $R^a$は$F$を生成する。
        \item $a, b \in R^a$について$s_a(b) \in R^a$である。
        \item $a, b \in R^a$について$(a^\vee, b) \in \bZ$である。
        \item $W^a = \langle s_a \mid a \in R^a \rangle$の$V$への作用は固有不連続である。
    \end{enumerate}
    $W$を$R^a$のWeyl群という。
\end{definition}

\begin{theorem}
    $R$が$V$上の既約有限ルート系ならば、
    \begin{equation}
        R^a = \{\alpha + n \delta \in F \mid \alpha \in R, n \in \bZ\}
    \end{equation}
    は$V$上のアファインルート系である。
\end{theorem}
\begin{proof}
    やるだけ。
\end{proof}
\begin{remark}
    ここでのルートはKac--Moody代数の文脈における実ルートに対応する。虚ルートは$\delta$の定数倍である。
\end{remark}
\begin{proposition}\label{prop:structure-of-affine-weyl-group}
    $R$を既約有限ルート系とし、$R^a$を付随するアファインルート系とする。このとき$W^a = W \ltimes t(Q^\vee)$である。ただし$t(Q^\vee)$は$Q^\vee$のベクトルによる平行移動のなす部分群である。
\end{proposition}
\begin{proof}
    $\alpha \in R$とし$a = \alpha + \delta \in R^a$とすると
    \begin{equation}
        s_\alpha s_a(x) = s_\alpha(x - ((x, a) + 1) \alpha^\vee) = x - (x, \alpha)\alpha^\vee + ((x, a) + 1) \alpha^\vee = x + \alpha^\vee
    \end{equation}
    だから$t(Q^\vee) \subset W^a$である。また計算により$wt(\alpha^\vee)w^{-1} = t(w(\alpha^\vee))$がわかるから、$t(Q^\vee)$は正規部分群である。残りは自明である。
\end{proof}
\begin{definition}
    $W \ltimes t(Q^\vee)$を$R$に付随するアファインWeyl群という。命題\ref{prop:structure-of-affine-weyl-group}より、これは$R^a$のWeyl群と同一視できる。
\end{definition}
\begin{definition}
    \begin{enumerate}
        \item  $W^{ae} = W \ltimes t(P^\vee)$を$R$に付随する拡大アファインWeyl群という。これは自然に$V$にアファイン変換として作用する。
        \item $\Omega = \{w \in W^{ae} \mid l(w) = 0\} = \{w \in W \mid wA = A\}$を、基本アルコーブ$A$を保つ$w$たちのなす部分群とする。
    \end{enumerate}
\end{definition}
\begin{proposition}
    \begin{enumerate}
        \item $W^{ae} = \Omega \ltimes W^a$
        \item $\Omega \cong P^\vee/Q^\vee$
    \end{enumerate}
\end{proposition}
\begin{proposition}
    $B^{ae} = \Omega \ltimes B^a$である。
\end{proposition}
\begin{corollary}
    $B^{ae}$は次の2通りの表示をもつ。
    \begin{enumerate}
        \item (Coxeter表示) $B^{ae} = \langle T_0, \dots, T_n, \Omega\rangle$
        \item (Bernstein表示) $B^{ae} = \langle T_1, \dots, T_n, Y^{P^\vee}\rangle$
    \end{enumerate}
\end{corollary}
\begin{example}[$D_4$型の場合]
    \begin{itemize}
        \item  $V = \bR^4 = \bigoplus_{i=1}^4 \bR e_i$を標準的な内積でユークリッド空間とする。
        \item  $R = \{\pm e_i \pm e_j \mid 1 \leq i < j \leq 4\}$を$D_4$型の既約有限ルート系とする。
        \item $\alpha_1 = e_1 - e_2, \alpha_2 = e_2 - e_3, \alpha_3 = e_3 - e_4, \alpha_4 = e_3 + e_4$がルート系の基底(のひとつ)をなす。
        \item 最高ルートは$\theta = e_1 + e_2 = \alpha_1 + 2\alpha_2 + \alpha_3 + \alpha_4$である。
        \item $Q = Q^\vee = \bigoplus_{i=1}^4 \bZ e_i = \{(x_1, x_2, x_3, x_4) \in \bZ^4\mid \sum_i x_i \in 2 \bZ\}$である。
        \item $\omega_1 = e_1, \omega_2 = e_1 + e_2, \omega_3 = \frac{1}{2}(e_1 + e_2 + e_3 - e_4), \omega_4 = \frac{1}{2}(e_1 + e_2 + e_3 + e_4)$が基本ウェイトで、$P = P^\vee = \bigoplus_{i=1}^4 \bZ \omega_i$である。
        \item $s_i$ を$\alpha_i$に付随する鏡映とする。$s_1, s_2, s_3 \in W$はそれぞれ$e_1$と$e_2$、$e_2$と$e_3$、$e_3$と$e_4$を交換することで$V$に作用する。$s_4$は$e_3$を$-e_4$に、$e_4$を$-e_3$に送る。これらにより$W$は生成され、座標の入れ替えおよび偶数個の座標を$-1$倍するように$V$に忠実に作用し、$S_4 \ltimes (\bZ/2\bZ)^3$と同型である。
        \item $\alpha_0 = -\theta + \delta$
        \item $A = \{v \in V \mid (\alpha_i, v)>0, (\theta, v)<1\}$が基本アルコーブである。その境界は$n+1 = 5$個のアファイン超平面$\alpha_i = 0$である。
        \item $\Omega = \{1, \pi_1, \pi_3, \pi_4\} \cong \bZ/2\bZ \oplus \bZ/2\bZ$は$A$を保つ$W^{ae}$の部分群であり、$A$の境界の超平面と単純ルートの集合$\{\alpha_0, \dots, \alpha_4\}$を同一視することで単純ルートの集合に作用する。さらに単純ルートの集合を$\widetilde{D_4}$型Dynkin図形の頂点と同一視することで、$\Omega$は$\widetilde{D_4}$型Dynkin図形の自己同型群と同一視できる。これにより$\pi_i \in \Omega$はDynkin図形の頂点$0$と$i$を入れ替えるような折り返しによる自己同型に対応する。
        \item $W^{ae} = \Omega \ltimes W^a$は共役作用$\pi_i s_{\alpha_j} \pi_i^{-1} = s_{\pi_i(\alpha_j)}$により記述される。つまり$\Omega$の単純ルートの集合への作用と整合的である。
        \item 同様に$B^{ae} = \Omega \ltimes B^a$は共役作用$\pi_i T_{s_{\alpha_j}} \pi_i^{-1} = T_{s_{\pi_i(\alpha_j)}}$により記述される。
    \end{itemize}
\end{example}
\section{二重アファインブレイド群}
$\tau =\{\tau_0, \dots, \tau_n\}$を不定元の集合とする。ただし$s_i$と$s_j$が$W^a$の中で共役なときに$\tau_i = \tau_j$であると定める。そして$\bC_\tau = \bC(\tau_0, \dots, \tau_n)$を$\bC$上の有理関数体とする。

次に$e \in \bZ>0$を$(P, P^\vee) = \frac{1}{e}\bZ$となるような整数と定め、格子$P \oplus \bZ \frac{\delta}{e} \subset F = \Aff(V)$への$W^{ae}$の作用を
$w = t(\lambda)v \in W^{ae}, \lambda \in P^\vee, v \in W$と$\mu \in P, k \in \bZ$について
\begin{equation}
    w\left(\mu + \frac{k}{e} \delta\right) = v(\mu) +\left(k/e - (\lambda, \mu)\right)\delta
\end{equation}
あるいは$\mu + \frac{k}{e} \delta$を乗法的に$q^{k/e} X^\mu$と書いて
\begin{equation}
    w(q^{k/e}X^\mu) = q^{k/e} \cdot q^{-(\lambda, v(\mu))}X^{v(\mu)}
\end{equation}
と定義する。$v$の線形性よりこれはwell-definedである。これを用いて二重アファインブレイド群を以下で定義する。
\begin{definition}
    $B^{ae} = \langle T_0, \dots, T_n, \Omega\rangle$をCoxeter表示とする。このとき二重アファインブレイド群$B^d$は$T_0, \dots, T_n, \Omega, X^P, q^{1/e}$を生成元とし以下の関係式で定まるような群である。
    \begin{enumerate}
        \item $B^{ae}$のCoxeter表示の関係式
        \item $X^P$の交換関係式
        \item $q^{1/e}$は中心元
        \item $\alpha_0^\vee = -\theta^\vee$とし$i = 0, \dots, n$について
              \begin{itemize}
                  \item $(\mu, \alpha_i^\vee) = 0$のとき$T_i$と$X^\mu$は可換
                  \item $(\mu, \alpha_i^\vee) = 1$のとき$T_i$と$X^\mu$は$T_iX^\mu T_i = X^{s_i(\mu)}$を満たす
              \end{itemize}
        \item $\pi \in \Omega$について$\pi X^\mu \pi^{-1} = X^{\pi(\mu)}$
    \end{enumerate}
\end{definition}
\begin{example}[$D_4$型の場合]
    \begin{itemize}
        \item $\omega_1 = e_1, \omega_2 = e_1 + e_2, \omega_3 = \frac{1}{2}(e_1 + e_2 + e_3 - e_4), \omega_4 = \frac{1}{2}(e_1 + e_2 + e_3 + e_4)$を基本ウェイトとして$P = P^\vee = \bZ \omega_1 \oplus \bZ \omega_2 \oplus \bZ \omega_3 \oplus \bZ \omega_4$である。
        \item $e = 2$
        \item $\mu = \sum_i \xi_i \omega_i \in P$について$(\mu, \alpha_i^\vee) = 0$となる条件は$i=0$のとき$\xi_1 + 2\xi_2 + \xi_3 + \xi_4 = 0$、$i > 0$のとき$\xi_i = 0$である。
        \item $(\mu, \alpha_i^\vee) = 1$となる条件は$i=0$のとき$\xi_1 + 2\xi_2 + \xi_3 + \xi_4 = -1$、$i > 0$のとき$\xi_i = 1$である。
    \end{itemize}
    これらを踏まえて$B^d$の関係式を($X_i = X^{\omega_i}$として)具体的に書き下すと
    \begin{itemize}
        \item $B^{ae}$のCoxeter表示の関係式、$X^P$の交換関係式、$q^{1/e}$は中心元
        \item $1\leq i, j \leq 4$かつ$i \neq j$のとき$T_i$と$X_j$は可換
        \item $\mu = \sum_i \xi_i \omega_i$が$\xi_1 + 2\xi_2 + \xi_3 + \xi_4 = 0$を満たすとき$T_0$と$X^\mu$は可換
        \item $T_1X_1T_1 = X^{s_1(\omega_1)} = X^{\omega_2-\omega_1} = X_1^{-1}X_2$
        \item $T_2X_2T_2 = X^{s_2(\omega_2)} = X_1X_2^{-1}X_3X_4$
        \item $T_3X_3T_3 = X^{s_3(\omega_3)} = X^{\omega_2-\omega_3} = X_2X_3^{-1}$
        \item $T_4X_4T_4 = X^{s_4(\omega_4)} = X^{\omega_2-\omega_4} = X_2X_4^{-1}$
        \item $T_0X_1^{-1}T_0 = X^{s_0(\omega_1)} = q^{-1}X_1^{-1}X_2$
    \end{itemize}
\end{example}
\section{小平ファイバーの導来圏}

\begin{proposition}[修正が必要!]
    $X$をアファイン$D_n$型Dynkin図形に対応する小平ファイバーとする。
    既約成分を$G_0, \dots, G_n$とし、
    %既約成分のうち$G_0, G_1, G_{n-1}, G_n$が重複度$1$で他の既約成分が重複度$2$で現れるように番号づけしておく。
    $G_i \setminus \bigcup_{j \neq i}G_j$の閉点$x_i$を任意に選ぶ。
    さらに$x_i$に台を持つlci zero cycle $Z_i$を任意に選ぶ。
    $\Phi \in \Auteq D^b(X)$とし、$\Phi$は積分関手であると仮定する。
    このとき$\Phi(\cO_X) \cong \cO_X$かつ全ての$i$について$\Phi(\cO_{Z_i}) \cong \cO_{Z_i}$ならば、ある$f \in \Aut(X)$が存在して$\Phi \cong f^*$である。

    さらに$\Phi$が無限個の閉点の構造層を保つならば、$\Phi = \id$である。
\end{proposition}
\begin{proof}
    $L = \cO_X(Z_0 + \cdots + Z_n)$は豊富な直線束である。(中井--Moishezonの判定法。非常に豊富なことまで言える?)
    すると
    \begin{equation}\label{eq:exact-triangle-of-very-ample-line-bundle}
        L^{-1} \to \cO_X \xrightarrow{p} \cO_{Z_0} \oplus \cdots \oplus \cO_{Z_n}\xrightarrow{+1}
    \end{equation}
    という$D^b(X)$の完全三角形がある。$\Phi$は$\cO$と$\cO_{Z_i}$を保ち完全三角形を保つから、以下の図式がある。
    \[
        \begin{tikzcd}
            L^{-1} \ar[r] & \cO_X \ar[r, "p"] \ar[d, "\sim", sloped] & \cO_{Z_0} \oplus \cdots \oplus \cO_{Z_n} \ar[r, "+1"]\ar[d, "\sim", sloped] &{}\\
            \Phi(L^{-1}) \ar[r] & \Phi(\cO_X) \ar[r, "\Phi(p)"] & \Phi(\cO_{Z_0}) \oplus \cdots \oplus \Phi(\cO_{Z_n}) \ar[r, "+1"] &{}\\
        \end{tikzcd}
    \]
    右側の縦の射を成分ごとに適切に定数倍で置き換えることで、これを可換図式にすることができる。よって$\Phi(L^{-1}) \cong L^{-1}$が成り立つ。$L$や一般の$L^m$についてもやはり\cite{MR3182005}と同様に$\Phi$で保たれることがわかる。

    よって$\Phi$は次数付きベクトル空間の同型
    \begin{equation}
        \varphi \colon \bigoplus_{m = 0}^\infty H^0(X, L^m) \xrightarrow{\sim} \bigoplus_{m = 0}^\infty H^0(X, \Phi(L^m)) \cong \bigoplus_{m = 0}^\infty H^0(X, L^m)
    \end{equation}
    を誘導する。Bondal--Orlov reconstructionの証明\cite{MR1818984}と同様に、2つ目の同型射を適切に取り替えることで$\varphi$は次数付き代数の準同型にできる。(使うのは$\Phi$が関手だから射の合成を保つことと、上の環の積が射の合成で定義されることである。)そこで$\varphi$の定める$X$の自己同型
    \begin{equation}
        X \cong \Proj(\bigoplus_{m = 0}^\infty H^0(X, L^m)) \xrightarrow{\varphi^*} \Proj(\bigoplus_{m = 0}^\infty H^0(X, L^m)) \cong X
    \end{equation}
    を$f$とおく。すると$\Phi \cong f^*$である。($f_*$かも。)

    $m$を十分大きくとって$L^m$が非常に豊富になるようにし、これ改めて$L$とおく。以下この$f$が恒等写像であることを示す。まず$f$は$L$の定める埋め込み$X \hookrightarrow \bP(H^0(X, L)^*)$により射影空間の自己同型$\widetilde{f}$に延長される。
    \[
        \begin{tikzcd}
            X \ar[r, hook] \ar[d, "f"] & \bP(H^0(X, L)^\vee) \ar[d, "\widetilde{f}"] \\
            X \ar[r, hook] & \bP(H^0(X, L)^\vee)
        \end{tikzcd}
    \]
    特に$\widetilde{f}$は線型写像で、$X$を含む最小の線形部分空間$V$を保つ。よって$\Phi$が$X$の無限個の閉点の構造層を保つならば$\widetilde{f}$は$V$上$\id$となり、特に$f$は$\id$となる。

    ちなみに$L$が非常に豊富でまで言えたとすると、無限個の閉点についての仮定なしに$\Phi = \id$が言える。以下のようにする。

    まずRiemann--Rochより
    \begin{equation}
        \chi(L) - \chi(\cO_X) = \sum_{i=0}^n \deg_{G_{i,\red}}(L) \cdot \length_{\eta_i}(\cO_{\eta_i})
    \end{equation}
    が成り立つ。ここで$\eta_i$は$G_i$の生成点である。
    さらに完全列\eqref{eq:exact-triangle-of-very-ample-line-bundle}より長完全列
    \begin{equation}
        0 \to H^0(L^{-1}) \to H^0(\cO_X) \to \bigoplus_{i=0}^n H^0(\cO_{Z_i}) \to \cdots
    \end{equation}
    があり、$\bC \cong H^0(\cO_X) \to \bigoplus_{i=0}^n H^0(\cO_{Z_i})$は$0$でない定数関数を$0$でない元に送るから単射である。
    よって(Serre双対から)$H^0(L^{-1}) \cong H^1(L)^\vee = 0$となる。
    以上を合わせると
    \begin{equation}
        \dim H^0(L) = 1 + 1 + 2(n-3) + 1 + 1 = 2n-2
    \end{equation}
    だから、$X$は$\bP^{2n-3}$に埋め込まれる。
    よって$X_{\red}$を含む最小の線形部分空間$V$は$\bP^{m}$($m \leq 2n-3$)と同型である。$V$は$x_0, \dots, x_n$と既約成分の交点$\{s_j\}_{j \in J}$($|J| = n$)でも生成されるが、$\widetilde{f}$はこれらの$2n+1 >2n-3$個の点を動かさない。よって$\widetilde{f}$は$V$上$\id$でなければならない。
    % \cite{MR3182005}のLemma 3.3と同様の議論をする。アファイン$A$型Dynkin図形に対応する小平ファイバーについては\cite{MR3182005}ですでに示されているので、$D$と$E$の場合を確かめる。

    % $D$型の場合、既約成分のうち$G_0, G_1, G_{n-1}, G_n$が重複度$1$で他の既約成分が重複度$2$で現れるように番号づけしておく。$X$上の非常に豊富な直線束$L$を
    % \begin{equation}
    %     L^{-1} = \Ker(\cO_X \to \cO_{x_0}\oplus \cO_{x_1} \oplus E_2 \oplus \cdots \oplus E_{n-2}\oplus \cO_{x_{n-1}} \oplus \cO_{x_n})
    % \end{equation}
    % となるようにとる。ただし$E_i$は$x_i$を台に持つ長さ$2$のArtin加群である。
    % このとき$\Phi(E_i) \cong E_i$が成り立つ。なぜなら$E_i$は$\cO_{x_i}$を$\cO_{x_i}$で拡大することで得られるが、
    % \begin{equation}
    %     \Ext_X^1(\cO_{x_i}, \cO_{x_i}) \cong \Hom(\cO_{x_i}, \cO_{x_i})^* \cong \bC
    % \end{equation}
    % より$E_i$はそのような加群として同型をのぞいて一意に定まるからである。($\cO_{x_i}$は具体的に射影分解を作ることでperfectだとわかる。)
    % すると$\Phi$は$L^{-1}$を保つ。あとは\cite{MR3182005}と同様。

\end{proof}

\begin{definition}
    $\pi \colon S \to C$を楕円曲面とし、$C$上の相対Fourier--向井変換全体のなす群を$\FM_C(D^b(X))$とする。$X$を$S$のファイバーとし、$E \in D^b(X)$を$i \colon X \to S$で押し出したとき$D^b(S)$の球面対象になるようなものとする。このとき捻り関手$T_{i_*E} \in \Auteq D^b(S)$は$\FM_C(D^b(S))$の元となる(\cite{2023arXiv230212501A})。自然な群準同型
    \begin{equation}
        \FM_C(D^b(S)) \to \Auteq D^b(X)
    \end{equation}
    による$T_{i_*E}$の像を$H_E$と書き、$E$に付随する半捻り関手と呼ぶ。
\end{definition}
\begin{lemma}\label{lem:conjugate-action}
    $X$を既約でない小平ファイバーとし、$G$を既約成分の1つとする。また$\Phi \in \Aut(X) \ltimes \Pic(X) \subset \Auteq D^b(X)$とする。このとき
    \begin{equation}
        \Phi \circ H_{\cO_G(a)}\circ \Phi^{-1} = H_{\Phi(\cO_G(a))}
    \end{equation}
    が成り立つ。
\end{lemma}
\begin{proof}
    一番ややこしい。明らかに正しそうだし一般論を整備すれば簡単に示せるのかもしれないが、現状だと泥臭い計算をやるしかなさそう。
\end{proof}

\begin{lemma}\label{lem:special-line-bundles}
    $X$を既約でない小平ファイバーとし、$G_i$たちを既約成分とする。このとき$X$上の直線束$L_i$たちで次の条件を満たすものが存在する。
    \begin{enumerate}
        \item $L_i$は$G_i$上で次数$1$、他の既約成分上で次数$0$を持つ。
        \item $X \subset S$を$X$を特異ファイバーにもつ楕円曲面とし、$\cO(G_i) \in \Pic(S)$としたとき
              \begin{equation}
                  \cO(G_i)\vert_X \cong \bigotimes_j L_j^{\otimes (G_j\cdot G_i)}
              \end{equation}
              が成り立つ。
    \end{enumerate}
\end{lemma}
\begin{proof}
    $\Pic(X)$を具体的に表示すれば線形代数$+\alpha$ぐらいの問題になるはず。
\end{proof}
\begin{lemma}[{\cite{MR2198807}, Lemma 4.15}]\label{lem:composition-of-twists}
    $S$が非特異曲面で$G \subset S$が$(-2)$曲線、$a \in \bZ$のとき、
    \begin{equation}
        T_{\cO_G(a-1)}\circ T_{\cO_G(a)} \cong -\otimes \cO_S(G)
    \end{equation}
    が成り立つ。
\end{lemma}
\begin{lemma}
    $X$を楕円曲面$S$の既約でない小平ファイバーとし、$L_i$を$i$番目の既約成分$G_i$上で次数$1$、他の既約成分上で次数$0$をもつ任意の直線束とする。また$H_i = H_{\cO_{G_i}(-1)} \in \Auteq D^b(X)$とする。(補題\ref{lem:conjugate-action}が正しいと仮定すると)このとき全ての$i$について
    \begin{equation}
        H_iL_iH_i = -\otimes \cO_S(G_i)\vert_Y \otimes L_i
    \end{equation}
    が成り立つ。
\end{lemma}
\begin{proof}
    補題\ref{lem:conjugate-action}より
    \begin{equation}
        L_i^{-1}H_iL_i = H_{\cO_{G_i}(-2)}
    \end{equation}
    が成り立つ。よって補題\ref{lem:composition-of-twists}より
    \begin{equation}
        L_i^{-1}H_iL_iH_i = H_{\cO_{G_i}(-2)}H_i = -\otimes \cO_S(G_i)\vert_X
    \end{equation}
    となる。
\end{proof}
\begin{lemma}\label{lem:omega_in_automorphisms}
    $X$を既約でない小平ファイバーとし、対応する型のルート系を考える。このとき(アファイン)Dynkin図形への作用が$\Omega$と同じになるような$X$の自己同型群の部分群が存在する。ただし$X$の自己同型のDynkin図形への作用は既約成分への作用によって定めるものとする。
\end{lemma}
\begin{proof}
    $D$型の場合
\end{proof}


\section{二重アファインブレイド群作用}

\begin{proposition}\label{prop:action-of-affine-braid-group}
    $X$を既約でない小平ファイバーとし、$G_0, \dots, G_n$を既約成分とする。さらに$\cO_{G_i}(-1)$の定める半ねじり関手を$H_{\cO_{G_i}(-1)} = H_i \in \Auteq D^b(X)$とする。また$B^a$を$X$に対応するアファインDynkin図形に付随するアファインブレイド群とする。このとき$H_0, \dots, H_n$は$B^a$の$D^b(X)$への作用を定める。さらに補題\ref{lem:omega_in_automorphisms}の自己同型群の部分群と合わせることで$B^{ae}$の作用を定める。
\end{proposition}
\begin{proof}
    \cite{MR1831820, brav2010braid,nordskova2020faithful}の結果などから前半は示せる(はず)。後半は補題\ref{lem:omega_in_automorphisms}と$B^{ae} = \Omega \ltimes B^a$からわかる。
\end{proof}
\begin{theorem}
    $X$をアファイン$D_n$型のDynkin図形に付随する小平ファイバーとする。$B^d = \langle B^{ae}, X^P, q\rangle$を対応する型の二重アファインブレイド群とする。このとき$B^{ae}$を命題\ref{prop:action-of-affine-braid-group}の作用、$X^P$を直線束による捻りの作用、$q$を$\id$とするような$B^d$の$D^b(X)$への作用が存在する。
\end{theorem}
\begin{example}
    \begin{itemize}
        \item $B^{ae}$のCoxeter表示の関係式、$X^P$の交換関係式、$q^{1/e}$は中心元
        \item $1\leq i, j \leq 4$かつ$i \neq j$のとき$T_i$と$X_j$は可換
        \item $\mu = \sum_i \xi_i \omega_i$が$\xi_1 + 2\xi_2 + \xi_3 + \xi_4 = 0$を満たすとき$T_0$と$X^\mu$は可換
        \item $T_1X_1T_1 = X^{s_1(\omega_1)} = X^{\omega_2-\omega_1} = X_1^{-1}X_2$
        \item $T_2X_2T_2 = X^{s_2(\omega_2)} = X_1X_2^{-1}X_3X_4$
        \item $T_3X_3T_3 = X^{s_3(\omega_3)} = X^{\omega_2-\omega_3} = X_2X_3^{-1}$
        \item $T_4X_4T_4 = X^{s_4(\omega_4)} = X^{\omega_2-\omega_4} = X_2X_4^{-1}$
        \item $T_0X_1^{-1}T_0 = X^{s_0(\omega_1)} = q^{-1}X_1^{-1}X_2$
    \end{itemize}
    以上の関係式を満たすように$X_i$の行き先を決めればよい。
    まず$X$上の直線束$L_0, \dots, L_4$を次の条件を満たすようにとる。
    \begin{enumerate}
        \item $L_i$は$G_i$上で次数$1$、他の既約成分上で次数$0$を持つ。
        \item $X \subset S$を$X$を特異ファイバーにもつ楕円曲面とし、$\cO(G_i) \in \Pic(S)$としたとき
              \begin{equation}
                  \cO(G_i)\vert_X \cong \bigotimes_{j=0}^4 L_j^{\otimes (G_j\cdot G_i)}
              \end{equation}
              が成り立つ。
    \end{enumerate}
    そして$X_i$の行き先を$i \neq 2$のとき$L_0^{-1}L_i \otimes-$、$i=2$のとき$L_0^{-2}L_2 \otimes -$とする。
\end{example}
\printbibliography[title=参考文献]
\end{document}