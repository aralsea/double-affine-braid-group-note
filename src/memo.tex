\RequirePackage{plautopatch}
\documentclass[uplatex, a4paper, dvipdfmx]{jsarticle}
\usepackage[backend=biber, style=alphabetic, sorting=nyt]{biblatex}
\addbibresource{../bib/double-affine-braid-group.bib} 

\usepackage{amsthm}
\usepackage{amsmath,amsfonts,amssymb}
\usepackage{tikz-cd}
\usepackage{dynkin-diagrams}

\renewcommand{\labelenumi}{(\arabic{enumi})}

\theoremstyle{definition}
\newtheorem{theorem}{定理}[section]
\newtheorem{definition}[theorem]{定義}
\newtheorem{proposition}[theorem]{命題}
\newtheorem{lemma}[theorem]{補題}
\newtheorem{corollary}[theorem]{系}
\newtheorem{fact}[theorem]{事実}
\newtheorem{example}[theorem]{例}
\newtheorem{remark}[theorem]{補足}

\DeclareMathOperator{\Hom}{\mathrm{Hom}}
\DeclareMathOperator{\Tor}{\mathrm{Tor}}
\DeclareMathOperator{\CHom}{\mathcal{H}\!\mathit{om}}
\DeclareMathOperator{\CTor}{\mathcal{T}\!\mathit{or}}
\DeclareMathOperator{\Auteq}{\mathrm{Auteq}}
\DeclareMathOperator{\Cone}{\mathrm{Cone}}
\DeclareMathOperator{\ev}{\mathrm{ev}}
\DeclareMathOperator{\id}{\mathrm{id}}
\DeclareMathOperator{\depth}{\mathrm{depth}}
\DeclareMathOperator{\Pic}{\mathrm{Pic}}
\DeclareMathOperator{\MCG}{\mathrm{MCG}}
\DeclareMathOperator{\PMCG}{\mathrm{PMCG}}
\DeclareMathOperator{\RHom}{\mathrm{RHom}}
\DeclareMathOperator{\Ker}{\mathrm{Ker}}
\DeclareMathOperator{\Coker}{\mathrm{Coker}}
\DeclareMathOperator{\Image}{\mathrm{Im}}
\DeclareMathOperator{\Aut}{\mathrm{Aut}}
\DeclareMathOperator{\Inn}{\mathrm{Inn}}
\DeclareMathOperator{\Out}{\mathrm{Out}}
\DeclareMathOperator{\Supp}{\mathrm{Supp}}
\DeclareMathOperator{\SL}{\mathrm{SL}}
\DeclareMathOperator{\GL}{\mathrm{GL}}
\DeclareMathOperator{\Spec}{\mathrm{Spec}}
\DeclareMathOperator{\Proj}{\mathrm{Proj}}
\DeclareMathOperator{\Perf}{\mathrm{Perf}}
\DeclareMathOperator{\NS}{\mathrm{NS}}
\DeclareMathOperator{\Ext}{\mathrm{Ext}}
\DeclareMathOperator{\Hilb}{\mathrm{Hilb}}
\DeclareMathOperator{\res}{\mathrm{res}}
\DeclareMathOperator{\Ch}{\mathrm{Ch}}
\DeclareMathOperator{\coh}{\mathrm{coh}}
\DeclareMathOperator{\Ann}{\mathrm{Ann}}
\DeclareMathOperator{\Aff}{\mathrm{Aff}}
\DeclareMathOperator{\FM}{\mathrm{FM}}
\DeclareMathOperator{\red}{\mathrm{red}}
\DeclareMathOperator{\length}{\mathrm{length}}
\DeclareMathOperator{\Div}{\mathrm{Div}}
\DeclareMathOperator{\DivCl}{\mathrm{DivCl}}
\DeclareMathOperator{\Cl}{\mathrm{Cl}}
\DeclareMathOperator{\Sch}{\mathrm{Sch}}
\DeclareMathOperator{\Set}{\mathrm{Set}}
\DeclareMathOperator{\op}{\mathrm{op}}
\DeclareMathOperator{\Coh}{\mathrm{Coh}}
\DeclareMathOperator{\QCoh}{\mathrm{QCoh}}




\newcommand{\nc}{\newcommand}

%% Calligraphic letters

\nc{\cF}{{\mathcal{F}}}
\nc{\cG}{{\mathcal{G}}}
\nc{\cH}{{\mathcal{H}}}
\nc{\cJ}{{\mathcal{J}}}
\nc{\cM}{{\mathcal{M}}}
\nc{\cO}{{\mathcal{O}}}
\nc{\cU}{{\mathcal{U}}}
\nc{\cW}{{\mathcal{W}}}

%% Blackboard letters
\nc{\bA}{{\mathbb{A}}}
\nc{\bC}{{\mathbb{C}}}
\nc{\bG}{{\mathbb{G}}}
\nc{\bP}{{\mathbb{P}}}
\nc{\bQ}{{\mathbb{Q}}}
\nc{\bR}{{\mathbb{R}}}
\nc{\bZ}{{\mathbb{Z}}}


%% Fraktur letters
\nc{\fg}{{\mathfrak{g}}}
\nc{\fu}{{\mathfrak{u}}}

% hyperref
\usepackage[urlcolor=blue]{hyperref}


\title{メモ}
\author{荒井 勇人}
\date{\today}
\begin{document}
\maketitle

\begin{itemize}
    \item 体$k$上の代数的スキームとは、$k$上分離的かつ有限型なスキームのことである。
\end{itemize}

\begin{lemma}[{\cite[Lemma 4.3]{MR1651025}}]\label{lem:flatness}
    $\pi \colon S \to T$をNoetherスキームの間の平坦射とする。閉点$t \in T$にたいしファイバーを$i_t \colon S_t \hookrightarrow S$と書く。
    $E \in D_{\Coh}^b(\cO_S)$とし、任意の閉点$t \in T$について$Li_t^*E$が$S_t$上の連接層になるとすると、$E$は$T$上平坦な$S$の連接層である。
\end{lemma}
\begin{proof}
    スペクトル系列
    \begin{equation}
        E_2^{p,q} = L_{-p}i_t^*(H^q(E)) \Rightarrow L_{-(p+q)}i_t^*E
    \end{equation}
    を考える。仮定より右辺は$p + q \neq 0$のとき$0$である。$q_0$を$H^{q_0}(E) \neq 0$となる最大のものとする($E$は上に有界なのでこういうものが存在する)と、$E_2^{0, q_0}$はある$t \in T$についてスペクトル系列の中で消えずに残る。よって$q_0 = 0$となる必要がある。

    $E$が$T$上平坦でないと仮定する。
\end{proof}
% \begin{proposition}
%     $X$を体$k$上の準射影的スキームとする。$X _1 = X_2 = X$とし、$\pi_i \colon X_1 \times X_2 \to X_i$を第$i$成分への射影とする。
%     $P \in D_{\Coh}^b(\cO_{X_1 \times X_2})$とし、$\Phi = \Phi^P$とおく。
%     全ての閉点$x \in X$についてある閉点$y \in X$があり$\Phi(\cO_x) \cong \cO_y$を満たすとする。このときある$\varphi \colon X\to X$と$L \in \Pic(X)$があり$\Phi \cong \varphi_*(L \otimes -)$となる。
% \end{proposition}
% \begin{proof}
%     まず補題\ref{lem:flatness}より$P$は$X_1$上平坦な連接層である。
%     次に$\cO_X(1)$を$X$上の非常の豊富な直線束とすると、十分大きい$d$について
%     \begin{equation}\label{eq:surjection}
%         \pi_1^*\pi_{1*}(P \otimes \pi_2^*\cO_X(d)) \to P \otimes \pi_2^*\cO_X(d)
%     \end{equation}
%     は全射である。

%     さらに$\pi_{1*}(P \otimes \pi_2^*\cO_X(d))$は十分大きい$d$について$X$上の直線束となる。
%     これを$L$とおくと、\eqref{eq:surjection}は全射
%     \begin{equation}
%         \cO_{X_1 \times X_2} \to P \otimes \pi_{1*}L^\vee  \otimes \pi_2^*\cO_X(d)
%     \end{equation}
%     を誘導する。
%     ここで同型
%     \begin{equation}
%         (P \otimes \pi_{1*}L^\vee  \otimes \pi_2^*\cO_X(d))_{\{x\} \times X_2} \cong \cO_{(x, y)}
%     \end{equation}
%     がある。
% \end{proof}

\begin{proposition}
    $X$をNoetherスキームとし、
\end{proposition}
\printbibliography[title=参考文献]
\end{document}